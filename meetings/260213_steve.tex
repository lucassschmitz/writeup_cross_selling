%\documentclass[notes,10pt,aspectratio=169]{beamer}

%\documentclass[notes, 10pt,aspectratio=169]{beamer}
\documentclass[10pt,aspectratio=169]{beamer}


% Add this line to your preamble
%\setbeameroption{show notes on second screen=right}

%\usetheme{Singapore} %Boadilla, Madrid, default, etc. 
\usetheme[progressbar=frametitle]{metropolis}
\usecolortheme{rose} %beaver, dolphin, crane, 


%\setbeamersize{text margin left=4mm, text margin right=4mm}


\usecolortheme{default}

\usepackage[utf8]{inputenc}
\usepackage[T1]{fontenc}
\usepackage{lmodern}
\usepackage{xcolor}
\usepackage{tikz}
\usepackage{booktabs} % Required for \toprule, \midrule, \bottomrule
\usetikzlibrary{shapes.geometric, arrows, positioning}

\tikzstyle{block} = [rectangle, draw, text width=4cm, align=center, rounded corners, minimum height=1cm]
\tikzstyle{decision} = [rectangle, draw, text width=5cm, align=center, fill=blue!10, rounded corners, minimum height=1cm]
\tikzstyle{terminal} = [rectangle, draw, text width=4.5cm, align=center, fill=yellow!30, rounded corners, minimum height=1cm]
\tikzstyle{end} = [rectangle, draw, text width=5cm, align=center, fill=green!30, rounded corners, minimum height=1cm]
\tikzstyle{arrow} = [->, thick]



\usepackage{adjustbox}
%2. change the bullets 
\setbeamertemplate{itemize item}[triangle] %circle, square,... 


% 1. Define custom colors and set colors 
%\definecolor{myblue}{HTML}{003366}
\definecolor{accent}{RGB}{78,205,196}

%\setbeamercolor{title}{fg=white,bg=myblue}
\setbeamercolor{frametitle}{fg=black,bg=white}
%\setbeamercolor{normal text}{fg=mygray}
\setbeamercolor{block title}{fg=black,bg=blue}
%\setbeamercolor{block body}{fg=black,bg=white}

\setbeamercolor{item}{fg= orange!80} % Change bullet color
\setbeamercolor{button}{bg=orange, fg=white}





% 3. BibLaTeX settings
\usepackage[
  backend=biber,
  style=apa,
  citestyle=authoryear
]{biblatex}
\addbibresource{../references.bib}

\title{Meetin with SPoints to discuss with Steve}
%\subtitle{A Mini Literature Overview}

\author{%
 Lucas Condeza
\inst{1} \and
   %\and
%  Coauthor Three\inst{3}
}
\institute{
  \inst{1} Yale University \\
}

\date{\today}

\begin{document}

%\begin{frame}
%  \titlepage
%\end{frame}


\begin{frame}{Last meeting}
\textbf{Last meeting:}
\begin{itemize}
    \item Presented outline of a research idea about cross-selling 
\end{itemize}

\textbf{Today's meeting:}
\begin{itemize}
    \item Present model of multi-product banks with switching costs and asymmetric information 
    \item What patterns in the data would be consistent with asymmetric information 
    \item Recommendations on next steps considering that I still have to apply for the data 
\end{itemize}

\end{frame}


%-------------------------------
\begin{frame}{Model: Setup (Adapting Engelbrecht-Wiggans et al., 1983)}

\begin{itemize}
  \item Model of a banking duopoly 
  \item Two periods:
  \begin{itemize}
    \item $t =1$: borrower gets an introductory product
    \item $t=2$: borrower requests a loan 
  \end{itemize} 
  \item In $t=2$ the incumbent bank observes the borrower's default probability $h$, the entrant only the distribution $F(h)$ with pdf $f(h)$
  \item In $t=2$ the borrower incurs a switching cost $\lambda$ if switching to the entrant bank
\end{itemize}


\vspace{0.3cm}
\textbf{Strategies:}
\begin{itemize}
    \item Bank 1 (incumbent): $r_1(h) = \sigma(h)$, an increasing function of $h$
    \item Bank 2 (entrant): mixed strategy $G(x) = \Pr(r_2 \leq x)$
    \item Borrower chooses bank 2 if $r_1 > r_2 + \lambda$
\end{itemize}

\end{frame}

%-------------------------------
\begin{frame}{Model: Equilibrium}

\textbf{Expected profits:}
\begin{align*}
    \pi_1(r_1(h)) &= [1-G(\sigma(h)-\lambda)] \cdot [(1-h)\sigma(h)-1] \\
    \pi_2(r_2) &= \Pr(\sigma(h)> r_2 +\lambda)\cdot \left[E[(1-h)r_2 - 1 \mid \sigma(h) > r_2 + \lambda \right]
\end{align*}
where $\tau = \sigma^{-1}$ is the inverse of bank 1's strategy.

\vspace{0.2cm}
\textbf{Equilibrium strategies:}
\begin{itemize}
    \item Incumbent's optimal strategy:
    $$\sigma(h) = \lambda + \frac{1}{ E[1-H \mid H > h]}$$
    \item Bank 2's mixed strategy:
\end{itemize}
\begin{align*}
    G(r - \lambda) = 1 - \exp\left[-\int_{\underline{r}}^{r} \frac{1-\tau(u)}{(1-\tau(u))u - 1} du\right]
\end{align*}

\end{frame}

%-------------------------------
\begin{frame}{Observations}

\begin{itemize}
    \item Entrant faces adverse selection: switchers have higher $h$
    \item Informational asymmetries create rents due to market power in $t=2$, but they are competed away in $t=1$
    \begin{itemize}
      \item Information asymmetries can lead to higher financial inclusion
    \end{itemize}
\end{itemize}

\end{frame}

%-------------------------------
% SIMULATION RESULTS
%-------------------------------
\begin{frame}{Simulation: Switching Probability by Repayment}
    \centering
    \includegraphics[width=0.85\textwidth]{../figures/Simulations/Model4/switching_probability_by_repayment.png}
\end{frame}



%-------------------------------
% Plots
%-------------------------------
\begin{frame}{Mortgage default rates}
    \begin{figure}
        \centering
        \includegraphics[width=0.85\textwidth]{../figures/Screenshots/Marcel21_BC_2_cropped.png}
        \caption{\textcite{marcel_pdl_2021}}
    \end{figure}
\end{frame}

%\begin{frame}{Mortgage default rates}
%    \centering
%    \includegraphics[width=0.85\textwidth]{../figures/Screenshots/Marcel21_BC.png}
%\end{frame}

%-------------------------------
% IDENTIFICATION
%-------------------------------

\begin{frame}{Identification}

\textbf{Observables:} winning rate $r_i^{win}$, winner identity $W_i \in \{1,2\}$, default $D_i \in \{0,1\}$
\begin{itemize}
    \item \textbf{Switching cost $\lambda$:} Identified from the gap between the minimum incumbent rate and the minimum entrant rate: $\lambda = \min(r_1^{win}) - \min(r_2^{win})$
    
    \item Is the model identified? I recovered the parameters of a simulation, but do not have a formal proof. 
    
\end{itemize}

\end{frame}

%-------------------------------
\begin{frame}{Estimation: MLE}

Each observation is $(r_i^{win}, W_i, D_i)$ a rate, the winner identity and the default outcome. 



\begin{itemize}
    \item \textbf{Incumbent wins} ($W_i = 1$): type is revealed, $h_i = \tau(r_i^{win})$
    \begin{align*}
        \mathcal{L}_i = \underbrace{f(\tau(r_i^{win})) \cdot |\tau'(r_i^{win})|}_{\text{density of rate}} \cdot \underbrace{[1 - G(r_i^{win} - \lambda)]}_{\text{prob. incumbent wins}} \cdot \underbrace{h_i^{D_i}(1-h_i)^{1-D_i}}_{\text{default outcome}}
    \end{align*}
    
    \item \textbf{Entrant wins} ($W_i = 2$): type is unobserved, integrate out $h$
    
    \begin{align*}
    \mathcal{L}_i(r_i^{win}, D_i, W_i = 2; \theta) &= \int_{h_{min}}^{h_{max}} \Pr(r_i^{win}, D_i, W_i = 2 \mid h; \theta) \cdot f(h; \alpha, \beta) \, dh \\
    &= \underbrace{g(r_i^{win})}_{\text{density of entrant's offer}} \cdot \int_{\tau(r_i^{win} + \lambda)}^{h_{max}} \underbrace{h^{D_i}(1-h)^{1-D_i}}_{\text{default outcome}} \cdot \underbrace{f(h)}_{\text{type density}} \, dh
    \end{align*}
\end{itemize}

\end{frame}


\end{document}









































 