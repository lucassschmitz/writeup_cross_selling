\documentclass[12pt]{article}
%%%%%%%%%%%%%%%%%%%%%%%%%%%%%%%%%%%%%%%%%%%%%%%%%%%%%%%%%%%%%%%%%%%%%%%%%%%%%%%%%%%%%%%%%%%%%%%%%%%%%%%%%%%%%%%%%%%%%%%%%%%%%%%%%%%%%%%%%%%%%%%%%%%%%%%%%%%%%%%%%%%%%%%%%%%%%%%%%%%%%%%%%%%%%%%%%%%%%%%%%%%%%%%%%%%%%%%%%%%%%%%%%%%%%%%%%%%%%%%%%%%%%%%%%%%%
\usepackage{amsfonts}
\usepackage{eurosym}
\usepackage{geometry}
\usepackage{amsmath,amsthm,amssymb}
\usepackage{ulem} 
\usepackage{graphicx}
\usepackage{comment}
%\usepackage[sort,comma]{natbib}
\usepackage[utf8]{inputenc}
\usepackage{setspace}
\usepackage[backend=biber, style = apa]{biblatex}
\usepackage{placeins} % to separate sections

\usepackage{adjustbox}
\usepackage{array}
\usepackage{multirow}
\usepackage{graphicx}
\usepackage{subcaption}
\usepackage{pifont}
\usepackage{amssymb}
\usepackage{comment}
\usepackage[hang, flushmargin, bottom]{footmisc}
\usepackage{footnotebackref}
\usepackage{xcolor}
\usepackage{hyperref}
\usepackage{booktabs}
\usepackage{pifont}
\usepackage{caption}
\usepackage{float}
\usepackage{todonotes}
\setcounter{MaxMatrixCols}{10}


%\setlength{\bibsep}{0.3pt}
\setlength{\textfloatsep}{5pt}
\hypersetup{breaklinks=true,hypertexnames=false,colorlinks=true,citecolor = teal}
\captionsetup{font=normalsize}
\newcommand{\cmark}{\ding{51}}
\def\sym#1{\ifmmode^{#1}\else\(^{#1}\)\fi}
\renewcommand{\thetable}{\Roman{table}}
\geometry{verbose,tmargin=.9in,bmargin=1in,lmargin=.8in,rmargin=.8in,nomarginpar}
\makeatletter
\DeclareTextSymbolDefault{\textquotedbl}{T1}
\theoremstyle{plain}
\newtheorem{thm}{\protect\theoremname}
\theoremstyle{plain}
\newtheorem{prop}[thm]{\protect\propositionname}
\theoremstyle{definition}  % Add this line
\newtheorem{definition}[thm]{Definition}  % Add this line
\theoremstyle{remark}  % Add this line
\newtheorem{remark}[thm]{Remark}  % Add this line
\providecommand{\propositionname}{Proposition}
\providecommand{\theoremname}{Theorem}
\makeatother
\newtheorem{ass}[thm]{Assumption}
% \input{tcilatex}
\usepackage{tikz}
\usetikzlibrary{shapes.geometric, arrows, positioning}


\addbibresource{references.bib}
\begin{document}


\section{Introduction}
 

\subsection{Motivation}


\begin{itemize}
    \item \textbf{Cross-selling: } the increase on the probability of buying an additional time given by the first purchase. 
    \item  The economic effects of cross-selling on welfare are not well understood. 
    \item The effects are ambiguous (see \textcite{klemperer_competition_1995}), 
    
\end{itemize}
 
\subsection{This research}


\section{Literature}

\begin{itemize}
    \item Empirical literature 
    \begin{itemize}
    \item   \textcite{basten_cross-selling_2023}
    1. documetns cross-selling, consumers that already have a deposit account at a bank are more likely to take out  a loan from the same bank than comparable household,
    2. disentangle demand and supply complementarities, where the former refers to the household switching cost and the later to information obtained by the firm. They find that existing depositors pay a risk-adjusted loan premium, moreover they do not find evidence of better screening, providing evidence more consistent with the demand complementarities story. 
    3. tehy try to disentangle two sources of demand complementarities: unobserved persistent preferences and inactions. They use consumer moving between locations to identify them and find that stickiness seems to be the main driver. 
    

    \item \textcite{qi_big_2024} finds profits of non-loan products cross-subsidize loans, specifically non-loan products are more profitable than loans and when there is an exogenous shock to the profitability of non-loan products, banks decrease loan supply presumably due to the decrease in cross-subsidization incentives. 
    \end{itemize}

    \item Structural  
    \begin{itemize}
    \item   \textcite{egan_dynamic_2025}
    Estimates a model where consumers are inattentive and choice banks with an endogenous probability, hence banks price dynamically considering the rents not only from the current period but also from consumers. They find that that sleepiness increases markups. 

    \item \textcite{dube_switching_2009}has a structural paper where consumers each period consume a good, and there is path dependence given by switching costs. They are able to identify the role of switching costs and persistent heterogeneity in preferences. 
    
    \item \textcite{brown_why_2023}
    
    \end{itemize}

    \item Theoretical literature 
    
    \begin{itemize}
        \item Add-ons: Ellison 

        \textcite{ellison_model_2005,ellison_search_2009} study add-ons. The setting is a duopoly where each firm sells a low and high quality product. The prices for the low quality product are public, but to learn about the high quality prices the consumer has to visit the firm at a cost $s$. 
        Consumers type is two-dimensional, they are horizontally differentiated a la Hotelling and they also differ in their marginal utility of income. 


        The difference with add ons is that an add on can only be purchased if the main product is purchased, while in cross-selling the products can be purchased independently. Although if enough people buy them together we can still use this type of models (e.g. loan and loan-penalties or loan insurance. )


        \item Switching costs: Klemperer 
        \item Relationship learning: \textcite{sharpe_asymmetric_1990,von_thadden_asymmetric_2004}
         
        \item Relationship banking with market power: \textcite{petersen_effect_1995}. They study a two-period model where there are high type and low type borrowers, and there is moral hazard. \footnote{In the model the high type borrowers can choose a risky project which is socially inefficient.} They find that 1) higher market power allos more firms to borrow, this is because the first period is used to screen borrowers and the second period to extract rents. More market power increases payoffs of the second stage, therefore increase
        
        \item \textcite{stiglitz_credit_1981} is  a paper that argues that in the credit market the prices do not necessarily clear the market since a higher interest rate selects riskier borrowers, leading to adverse selection. 
         
    \end{itemize}






\end{itemize}






\section{Setting and Data}

\subsection{Data}

\textcolor{red}{what data sources could we have access to? }



\section{Empirical Evidence}

\begin{itemize}
    \item Document cross-selling patterns in the data. For example what share of consumers who open a bank accoutn end up getting a mortgage at the same bank? 
    
    \item At what price do old customers get loans when compared to new customers? is it higher or lower? in the model what would we expect? What is the difference in rates for the combinations of low-high types, new-old consumers\footnote{define low-high types by prior behavior}? this could allow us to determine the role of switching costs and information. 
    
    \item 

\end{itemize}



\section{Model}
 
     \begin{itemize}
        \item Multi-product firms and multiple period 
        \item Switching costs: there is a large literature on switching costs (e.g. \textcite{klemperer_competition_1995,farrell_chapter_2007}), but they mostly study the case of two firms, the challenge is to extend this to an oligopoly setting. 
        
        \item Relationship learning: firms learn about consumer risk type
    \end{itemize}

    \subsection{Model 1}

    Model of switching costs with differentiated products in a two-period setting. Very similar to the model by \textcite{dube_switching_2009}. 

     
    \subsection{Model 2}


\subsection{Model 3}

Another model that could be used is an extension of \textcite{allen_search_2019} where one could extend the model to include a first stage where firms compete for customers. 

\subsection{Model 4}

Another possibility is to just model the switching costs like \textcite{egan_dynamic_2025} and coauthors where the borrower is inattentive with certain probability. Possibly following their approach where inattentiveness depends on a market-specific and aggreate state variable. The problem of this is that it does not incorporate the value of information that the bank gets. 


If one combines the Sharpe-von Thadden model which is able to solve for the second stage with a model of inattention the model would be quite simple to solve and would provide an insight into information assymetries and switching costs. 

If doing this approach is important to clearly show what the model is adding from the standard pricing scenario, for example \textcite{egan_dynamic_2025} in equation (6) clearly show the FOC and how inattention affects the optimal price chosen by the bank (the harvesting incentive). 

\section{Identification}
Here I will write examples of what variation could help us to identify different aspects of the model. 
\begin{itemize}
    \item Take the case of an individual who has a checking account at bank A and B, but not at C. And he wants to get a mortgage. But currently he only uses the account from A, B is inactive. 
    Hence banks A and B have information about the customer, but C does not. [in progress]
\end{itemize}
\end{document}