\documentclass[10pt,aspectratio=169]{beamer}
%\setbeameroption{show notes on second screen=right}
\usetheme[progressbar=frametitle]{metropolis}
\usecolortheme{rose} %beaver, dolphin, crane, 


\setbeamersize{text margin left=4mm, text margin right=4mm}


\usecolortheme{default}
\setbeamertemplate{navigation symbols}{}
\setbeamertemplate{footline}[frame number]
\usepackage[utf8]{inputenc}
\usepackage[T1]{fontenc}

\usepackage{amsmath, amssymb, bm}
\usepackage{booktabs}
\usepackage{graphicx}
\usepackage{ragged2e}
\usepackage{hyperref}
%\hypersetup{colorlinks=true, urlcolor=blue}


\AtBeginSection[]
{
  \begin{frame}{Outline}
    \tableofcontents[currentsection]%,hideothersubsections]
  \end{frame}
}

\setbeamercolor{item}{fg= orange!80} % Change bullet color
\setbeamercolor{button}{bg=orange, fg=white}


%%%%%%%%%%%%%%%%%%%%%%%%%%%%%%%%%%%%%%%%%%%%%%%%%%%%%%%%%%%%%%%%%%%%%%%%%%%%%%%
% Title info
%%%%%%%%%%%%%%%%%%%%%%%%%%%%%%%%%%%%%%%%%%%%%%%%%%%%%%%%%%%%%%%%%%%%%%%%%%%%%%%

\title[Sleepy Deposits]{Dynamic Competition for Sleepy Deposits}
\subtitle{Mark L. Egan, Ali Horta\c{c}su, Nathan A. Kaplan, Adi Sunderam, Vincent Yao\\
NBER Working Paper 34267 (September 2025)}
\author[Lucas Schmitz]{Lucas Schmitz}
\institute{Yale Economics}
\date{\today}

%%%%%%%%%%%%%%%%%%%%%%%%%%%%%%%%%%%%%%%%%%%%%%%%%%%%%%%%%%%%%%%%%%%%%%%%%%%%%%%
\begin{document}
%%%%%%%%%%%%%%%%%%%%%%%%%%%%%%%%%%%%%%%%%%%%%%%%%%%%%%%%%%%%%%%%%%%%%%%%%%%%%%%

% Title page
\begin{frame}
  \titlepage
\end{frame}

%%%%%%%%%%%%%%%%%%%%%%%%%%%%%%%%%%%%%%%%%%%%%%%%%%%%%%%%%%%%%%%%%%%%%%%%%%%%%%%
\section{Motivation and Questions}
%%%%%%%%%%%%%%%%%%%%%%%%%%%%%%%%%%%%%%%%%%%%%%%%%%%%%%%%%%%%%%%%%%%%%%%%%%%%%%%

\begin{frame}{Motivation: ``Sleepy'' retail deposits}
Retail depositors rarely shop for better terms.
\begin{itemize}
	\item Account turnover is low: new checking/savings accounts are roughly 8--15\% of existing accounts per year (average life $\approx$ 8 years).
	\item Account closings are mostly idiosyncratic (e.g., inactivity, no longer needed, moving, death); only about 17\% of closures cite switching for better rates/services/fees.
	\item Estimated share inactive each year is very high ($\approx 94\%$).
\end{itemize}
\end{frame}

\begin{frame}{Why this matters}
Sleepiness changes the industrial organization of deposit markets.
\begin{itemize}
	\item Competition: does depositor inertia soften competition, or induce dynamic ``invest vs.\ harvest'' incentives that can also intensify competition for active depositors?
	\item Bank value: how much of the deposit franchise value is attributable to sleepy deposits?
	\item Financial stability: how does the deposit franchise buffer banks when rates rise?
	\item Policy relevance: regulators (e.g., UK FCA) have pushed for more pass-through of interest rates to depositors; what happens if deposit competition becomes effectively static?
\end{itemize}
\end{frame}

\begin{frame}{This paper: core questions}
\begin{itemize}
	\item How sleepy are retail depositors, and how does sleepiness vary with rates and demographics?
	\item What does sleepiness imply for deposit market markups and their cyclicality?
	\item How does a dynamic view change the relationship between concentration and markups?
	\item How large is the deposit franchise value effect, and who benefits most?
	\item What happens under counterfactuals that eliminate sleepiness / dynamic competition?
\end{itemize}
\end{frame}

%%%%%%%%%%%%%%%%%%%%%%%%%%%%%%%%%%%%%%%%%%%%%%%%%%%%%%%%%%%%%%%%%%%%%%%%%%%%%%%
\section{Data and Stylized Facts}
%%%%%%%%%%%%%%%%%%%%%%%%%%%%%%%%%%%%%%%%%%%%%%%%%%%%%%%%%%%%%%%%%%%%%%%%%%%%%%%

\begin{frame}{Data sources (and market definition)}
\begin{itemize}
	\item Account-level turnover microdata from a core account processing platform: 89 banks/credit unions, 12 million deposit accounts.
	\item FDIC Summary of Deposits (SoD), 1994--2023: branch deposits aggregated to bank$\times$county$\times$year.
	\item RateWatch (2001--June 2023): bank$\times$county$\times$year average rates by account type.
	\item Call Reports: bank-level cost shifters (salary expenses, premises/fixed-asset expenses).
	\item Final estimation panel: 341{,}395 bank$\times$county$\times$year observations, 2002--2023 (dropping single-bank markets).
\end{itemize}
\end{frame}

\begin{frame}{Deposit rates and spreads}
Paper’s baseline rate measure uses \textbf{\$10k money market deposit account} rates from RateWatch.
\begin{itemize}
	\item Deposit spread defined as $\rho_{jkt} := R_t^F - r_{jkt}$ (``price'' of deposit services).
	\item Summary stats (baseline sample):
	\begin{itemize}
		\item Average deposit rate $\approx$ 34 bps.
		\item Average spread $R_t^F - r_{jkt} \approx$ 97 bps.
		\item Average market share per bank$\times$county observation $\approx$ 16\%.
	\end{itemize}
\end{itemize}
\end{frame}

\begin{frame}{Stylized fact 1: low account openings (turnover)}
\begin{itemize}
	\item New checking/savings accounts are roughly 8--15\% of existing accounts per year (life $\approx$ 8 years).
	\item Time deposits turn over faster (natural maturity).
	\item Cross-sectional patterns suggest frictions (search/switching/inattention), not just persistent tastes:
	\begin{itemize}
		\item Higher turnover for business and large-balance accounts.
		\item Lower turnover for age 65+.
	\end{itemize}
\end{itemize}

\vspace{0.3em}
\centering
%\maybegraphics[width=0.85\textwidth]{Figures/fig_account_life.pdf}
\end{frame}

\begin{frame}{Stylized fact 2: why accounts close}
\begin{itemize}
	\item Banks record closure reasons for about 75\% of closures.
	\item Most common: \textit{inactivity} (bank closes on customer’s behalf).
	\item Moves and deaths account for $>$20\% of closures.
	\item Only about 17\% cite switching for better rates/services/fees.
\end{itemize}

\vspace{0.3em}
\centering
%\maybegraphics[width=0.85\textwidth]{Figures/fig_closure_reasons.pdf}
\end{frame}

%%%%%%%%%%%%%%%%%%%%%%%%%%%%%%%%%%%%%%%%%%%%%%%%%%%%%%%%%%%%%%%%%%%%%%%%%%%%%%%
\section{Model}
%%%%%%%%%%%%%%%%%%%%%%%%%%%%%%%%%%%%%%%%%%%%%%%%%%%%%%%%%%%%%%%%%%%%%%%%%%%%%%%

\begin{frame}{Model overview}
Dynamic deposit market with \textbf{active vs.\ inactive} depositors.
\begin{itemize}
	\item Each period depositors are either awake (active) or asleep (inactive).
	\item Active depositors choose a bank via discrete choice; inactive depositors keep last period’s bank.
	\item Banks set deposit rates/spreads dynamically to maximize franchise value.
	\item Sleepiness generates \textbf{invest vs.\ harvest} incentives:
	\begin{itemize}
		\item Harvest: inertia lowers elasticity $\Rightarrow$ lower offered rates (higher spreads).
		\item Invest: capturing an active depositor today raises future deposits (if they become inactive).
	\end{itemize}
\end{itemize}
\end{frame}

\begin{frame}{Depositor utility and demand among active}
Active depositor $i$ choosing bank $j$ in market $k$ at time $t$:
\[
u_{ijkt} = \alpha \rho_{jkt} + \delta_{jkt} + \varepsilon_{ijkt},
\qquad \rho_{jkt} := R_t^F - r_{jkt},\quad \alpha < 0.
\]
With Type-I extreme value $\varepsilon_{ijkt}$, active market share:
\[
s^{Active}_{jkt} = \frac{\exp(\alpha \rho_{jkt} + \delta_{jkt})}{\sum_{\ell\in J_k}\exp(\alpha \rho_{\ell kt} + \delta_{\ell kt})}.
\]
\end{frame}

\begin{frame}{Sleepiness (inactivity) process}
Depositor activity depends on observables and an idiosyncratic shock:
\[
D_{it}^* = S'_{k(i)t}\Gamma + X'_{it}\Theta + \eta_{it},
\qquad D_{it} = 1(D_{it}^*>0).
\]
Market-level inactivity share:
\[
\phi_{kt} = 1 - \mathbb{E}[D_{it}\mid S_{kt}].
\]
Interpretation: reduced-form inattention / switching frictions; empirically $\phi_{kt}$ varies with rates and demographics.
\end{frame}

\begin{frame}{Total deposits: active + inactive components}
Total deposits held by bank $j$ in market $k$ at time $t$:
\[
Dep_{jkt} = (1-\phi_{kt})M_{kt}s^{Active}_{jkt}
\;+\;
\phi_{kt}(1+r_{jk,t-1})Dep_{jk,t-1}.
\]
\begin{itemize}
	\item Active flow: $(1-\phi_{kt})M_{kt}s^{Active}_{jkt}$.
	\item Inactive stock: last period deposits roll over and earn interest.
\end{itemize}
\end{frame}

\begin{frame}{Banks: flow profits and franchise value}
Banks invest deposits at bank-level return $R_{jt}$ and pay deposit rate $r_{jkt}$ and marginal servicing cost $c_{jkt}$.
\[
\pi_{jt}=\sum_{k\in K} Dep_{jkt}\Big((R_{jt}-R_t^F)+\rho_{jkt}-c_{jkt}\Big),
\quad \rho_{jkt}=R_t^F-r_{jkt}.
\]
Franchise value:
\[
V_{jt}=\mathbb{E}\Big[\sum_{s=t}^{\infty}\beta^{\,s-t}\pi_{js}\Big].
\]
Equilibrium: stationary Markov perfect equilibrium in pure strategies (BBL approach).
\end{frame}

%%%%%%%%%%%%%%%%%%%%%%%%%%%%%%%%%%%%%%%%%%%%%%%%%%%%%%%%%%%%%%%%%%%%%%%%%%%%%%%
\section{Estimation Strategy}
%%%%%%%%%%%%%%%%%%%%%%%%%%%%%%%%%%%%%%%%%%%%%%%%%%%%%%%%%%%%%%%%%%%%%%%%%%%%%%%

\begin{frame}{Estimation: three steps}
\begin{enumerate}
	\item \textbf{Sleepiness} $\phi_{kt}$: combine account openings/turnover microdata and deposit autocorrelation in SoD using a control function.
	\item \textbf{Demand} (active depositors): compute active market shares using $\phi_{kt}$, estimate $(\alpha,\delta_{jkt})$ via Berry (1994).
	\item \textbf{Costs / supply} (dynamic game): estimate reduced-form spread policy functions; recover cost parameters via BBL (Bajari et al.\ 2007) using the condition that observed policies maximize franchise value.
\end{enumerate}
\end{frame}

\begin{frame}{Step 1: identifying sleepiness from account openings}
Under the model, new accounts identify active demand:
\[
NewDep_{\ell j t} = (1-\phi(\cdot))M_{\ell t}s^{Active}_{jt}(1-s_{j,t-1}).
\]
Rearrangement yields a moment linking turnover to sleepiness:
\[
1-\frac{Dep_{\ell j,t-1}}{Dep_{\ell jt}}\frac{NewDep_{\ell jt}}{1-s_{j,t-1}}
=
\phi(\cdot)
+
e_{\ell jt}.
\]
Key finding: $\approx 94\%$ of depositors are inactive each year; sleepiness falls when the lagged fed funds rate is higher (people ``wake up'' when returns to shopping rise).
\end{frame}

\begin{frame}{Step 1: autocorrelation and control function}
Autocorrelation in deposits confounds sleepiness with persistent quality.
\begin{itemize}
	\item Control function: regress spreads on cost shifters (salaries, fixed expenses) to recover latent demand:
	\[
	\rho_{jkt} = \lambda Z_{jt} + v_{jkt}.
	\]
	\item Second stage: regress deposits on lagged deposits and the control:
	\[
	Dep_{jkt} = (\Upsilon_1'S_{kt}+\Upsilon_2'X_{kt})(1+R_{t-1}^F-\rho_{jk,t-1})Dep_{jk,t-1}+H(v_{jkt})+\iota_{jkt}.
	\]
\end{itemize}
\end{frame}

\begin{frame}{Step 2: demand among active depositors}
\begin{itemize}
	\item Use estimated $\phi_{kt}$ to compute each bank’s \textit{active} market share $s^{Active}_{jkt}$.
	\item Estimate price sensitivity $\alpha$ and qualities $\delta_{jkt}$ using Berry (1994).
	\item Result: accounting for sleepiness makes deposit demand substantially more elastic (about 33--34\% higher elasticity relative to treating all depositors as active).
\end{itemize}
\end{frame}

\begin{frame}{Step 3: dynamic supply and costs (BBL)}
\begin{itemize}
	\item Parameterize (net) marginal costs:
	\[
	c_{jkt}=\omega+\zeta R_t^F+\gamma'Z_{jt}+\chi_{jkt}.
	\]
	\item Flexibly estimate policy function mapping states to spreads (first-order polynomial in states + competitor sums).
	\item Use BBL (Bajari et al.\ 2007): forward-simulate franchise values under the observed policy and many deviations; pick cost parameters that rationalize observed policy as optimal.
\end{itemize}
\end{frame}

%%%%%%%%%%%%%%%%%%%%%%%%%%%%%%%%%%%%%%%%%%%%%%%%%%%%%%%%%%%%%%%%%%%%%%%%%%%%%%%
\section{Main Results}
%%%%%%%%%%%%%%%%%%%%%%%%%%%%%%%%%%%%%%%%%%%%%%%%%%%%%%%%%%%%%%%%%%%%%%%%%%%%%%%

\begin{frame}{Result 1: sleepiness raises average markups}
Counterfactual: eliminate sleepiness (all depositors always active) $\Rightarrow$ static competition.
\begin{itemize}
	\item Average markups would be \textbf{53\% lower} without sleepiness:
	\begin{itemize}
		\item From about \textbf{68 bps} (baseline) to about \textbf{32 bps} (no sleepiness).
	\end{itemize}
\end{itemize}
\centering
%\maybegraphics[width=0.85\textwidth]{Figures/fig_markups_baseline_vs_nosleepy.pdf}
\end{frame}

\begin{frame}{Result 2: sleepiness makes markups procyclical}
\begin{itemize}
	\item With sleepiness, markups rise more when the short rate rises (invest-versus-harvest incentives vary over the cycle).
	\item Quantitatively:
	\begin{itemize}
		\item At the zero lower bound: average markup $\approx$ 31 bps.
		\item In 2023 (short rates $>$ 5\%): average markup $\approx$ 127 bps.
	\end{itemize}
\end{itemize}
\centering
%maybegraphics[width=0.85\textwidth]{Figures/fig_markup_cyclicality.pdf}
\end{frame}

\begin{frame}{Result 3: concentration–markup relationship changes in the dynamic model}
\begin{itemize}
	\item In static/reduced-form views, markups often appear increasing in concentration (HHI).
	\item Accounting for dynamic competition:
	\begin{itemize}
		\item Eliminates the increasing relationship between concentration and markups that obtains under static competition.
		\item Markups can be high even in low-HHI areas because banks endogenously harvest sleepy bases.
	\end{itemize}
\end{itemize}
\centering
%\maybegraphics[width=0.85\textwidth]{Figures/fig_hhi_markups_static_vs_dynamic.pdf}
\end{frame}

\begin{frame}{Result 4: sleepiness explains a large share of deposit franchise value}
\begin{itemize}
	\item Under the static counterfactual (no sleepiness), average deposit franchise value falls by about \textbf{58\%}.
	\item Interpretation: sleepiness accounts for more than half of the value of the average bank’s deposit franchise.
\end{itemize}
\centering
%\maybegraphics[width=0.85\textwidth]{Figures/fig_franchise_value_decomposition.pdf}
\end{frame}

\begin{frame}{Result 5: heterogeneity—who benefits most from sleepiness?}
Sleepiness dependence is larger for banks with low quality or high costs.
\begin{itemize}
	\item Moving from 25th to 75th percentile of product quality reduces sleepiness dependence by about 10.4 pp (62\% $\rightarrow$ 51.6\%).
	\item Moving from 25th to 75th percentile of marginal costs increases sleepiness dependence by about 3.4 pp (54.9\% $\rightarrow$ 58.3\%).
\end{itemize}
\centering
%\maybegraphics[width=0.85\textwidth]{Figures/fig_heterogeneity_quality_cost.pdf}
\end{frame}

\begin{frame}{Result 6: financial stability implications}
\begin{itemize}
	\item Convert franchise value losses into risk-neutral default probabilities (using CDS-based approach in the paper).
	\item For the three largest deposit-taking banks (JPM, BofA, Wells Fargo):
	\begin{itemize}
		\item Default probabilities rise by about 10 pp in normal times.
		\item Increase by \textbf{more than 20 pp} during the 2022--2023 tightening cycle absent sleepy deposits.
	\end{itemize}
\end{itemize}
\centering
%\maybegraphics[width=0.85\textwidth]{Figures/fig_default_prob_counterfactual.pdf}
\end{frame}

%%%%%%%%%%%%%%%%%%%%%%%%%%%%%%%%%%%%%%%%%%%%%%%%%%%%%%%%%%%%%%%%%%%%%%%%%%%%%%%
\section{Counterfactuals and Extensions}
%%%%%%%%%%%%%%%%%%%%%%%%%%%%%%%%%%%%%%%%%%%%%%%%%%%%%%%%%%%%%%%%%%%%%%%%%%%%%%%

\begin{frame}{Policy counterfactual: making deposit competition static}
Motivation: FCA and other regulators seek more pass-through to depositors (higher deposit betas).
\begin{itemize}
	\item Paper’s static counterfactual is a limiting case: constrain betas to unity $\Rightarrow$ banks compete on constant spreads rather than time-varying ones.
	\item Mechanism: eliminates dynamic invest-versus-harvest tradeoff induced by depositor sleepiness.
	\item Quantitative effects (from earlier results): lower markups, lower franchise value, higher fragility.
\end{itemize}
\end{frame}

\begin{frame}{Extension 1: price discrimination via bonuses}
Allow banks to offer different terms to new vs incumbent depositors.
\begin{itemize}
	\item Helps discriminate between active (new) and inactive (incumbent) depositors, but imperfectly.
	\item Estimated customer acquisition costs are large:
	\begin{itemize}
		\item Roughly \textbf{15 cents} to attract an additional \textbf{dollar} of deposits from new depositors.
	\end{itemize}
	\item Main conclusions about sleepiness and dynamic competition remain similar.
\end{itemize}
\end{frame}

\begin{frame}{Extension 2: forward-looking depositors}
Re-estimate demand when active depositors are forward-looking.
\begin{itemize}
	\item Adds an additional product characteristic: discounted expected utility of remaining asleep with today’s choice.
	\item Demand becomes more elastic relative to myopic baseline.
	\item Qualitative conclusions about sleepiness/markups/franchise value remain similar.
\end{itemize}
\end{frame}

%%%%%%%%%%%%%%%%%%%%%%%%%%%%%%%%%%%%%%%%%%%%%%%%%%%%%%%%%%%%%%%%%%%%%%%%%%%%%%%
\section{Conclusion}
%%%%%%%%%%%%%%%%%%%%%%%%%%%%%%%%%%%%%%%%%%%%%%%%%%%%%%%%%%%%%%%%%%%%%%%%%%%%%%%

\begin{frame}{Conclusion}
\begin{itemize}
	\item Retail depositors are extremely inactive; most switching is not active shopping.
	\item A dynamic model with sleepiness rationalizes invest-versus-harvest incentives.
	\item Sleepiness substantially raises average markups and makes them procyclical.
	\item Dynamic competition changes the mapping from concentration to markups.
	\item Sleepiness explains a large share of deposit franchise value and matters for stability.
\end{itemize}
\end{frame}

 

%%%%%%%%%%%%%%%%%%%%%%%%%%%%%%%%%%%%%%%%%%%%%%%%%%%%%%%%%%%%%%%%%%%%%%%%%%%%%%%
\appendix
%%%%%%%%%%%%%%%%%%%%%%%%%%%%%%%%%%%%%%%%%%%%%%%%%%%%%%%%%%%%%%%%%%%%%%%%%%%%%%%

\begin{frame}[plain,noframenumbering]
\centering
{\Large Appendix}
\end{frame}

\begin{frame}{Appendix A: extra figures (placeholders)}
\centering
%\maybegraphics[width=0.9\textwidth]{Figures/fig_appendix_placeholder_1.pdf}
\end{frame}
 

%%%%%%%%%%%%%%%%%%%%%%%%%%%%%%%%%%%%%%%%%%%%%%%%%%%%%%%%%%%%%%%%%%%%%%%%%%%%%%%
\end{document}
%%%%%%%%%%%%%%%%%%%%%%%%%%%%%%%%%%%%%%%%%%%%%%%%%%%%%%%%%%%%%%%%%%%%%%%%%%%%%%%
