\documentclass[12pt]{article}
\usepackage{amsfonts}
\usepackage{eurosym}
\usepackage{geometry}
\usepackage{amsmath,amsthm,amssymb}
\usepackage{graphicx}
\usepackage{comment}
\usepackage[utf8]{inputenc}
\usepackage{setspace}
%\usepackage[sort,comma]{natbib}
\usepackage[backend=biber, style = apa]{biblatex}
\usepackage{placeins} % to separate sections

\usepackage{adjustbox}
\usepackage{array}
\usepackage{multirow}
\usepackage{graphicx}
\usepackage{subcaption}
\usepackage{pifont}
\usepackage{amssymb}
\usepackage{comment}
 
\usepackage[hang, flushmargin, bottom]{footmisc}
\usepackage{hyperref}

\usepackage{footnotebackref}
\usepackage{xcolor}
\usepackage{booktabs}
\usepackage{pifont}
\usepackage{caption}
\usepackage{float}
\setlength{\marginparwidth}{2cm} 

\usepackage{todonotes}
\setcounter{MaxMatrixCols}{10}


%\setlength{\bibsep}{0.3pt}
\setlength{\textfloatsep}{5pt}
\hypersetup{breaklinks=true,hypertexnames=false,colorlinks=true,citecolor = teal}
\captionsetup{font=normalsize}
\newcommand{\cmark}{\ding{51}}
\def\sym#1{\ifmmode^{#1}\else\(^{#1}\)\fi}
\renewcommand{\thetable}{\Roman{table}}
\geometry{verbose,tmargin=.9in,bmargin=1in,lmargin=1in,rmargin=.9in,nomarginpar}
\makeatletter

\DeclareTextSymbolDefault{\textquotedbl}{T1}
\theoremstyle{plain}
\newtheorem{thm}{Theorem}%[section] commented out to avoid numbering by section
\newtheorem{prop}[thm]{Proposition}
\newtheorem{ass}[thm]{Assumption}
\newtheorem{lemma}[thm]{Lemma}
\newtheorem{theorem}[thm]{Theorem}   % alias for \begin{theorem}
\newtheorem{definition}{Definition}
\makeatother


\newcommand{\sepline}{\par\bigskip\noindent\rule{\linewidth}{0.4pt}\par\medskip}

% \input{tcilatex}
\usepackage{enumitem} % allows custom labels
\usepackage{tikz}
\usetikzlibrary{shapes.geometric, arrows, positioning}





\addbibresource{references.bib}
\begin{document}
 
% \title{{\Large Centralized annuities marketplace}}
%\author{Lucas Condeza\thanks{Yale University %\texttt{lucas.schmitz@yale.edu}}} 
%\date{}
%\maketitle


 

\begin{abstract}

In markets where prices are set for each individual  t is common for consumers to receive initial offers and then leverage them to request revised offers—improved quotes provided after the consumer asks for better terms-, yet there is limited evidence on their effects on market outcomes and welfare.


The effect of revised offers are ambiguous and depend on [USE THE MODEL TO EXPLAIN ON WHAT DO THEY DEPEND]

Using data that records initial and revised offers, and leveraging a reform that prohibited revised offers, we study the welfare and distributional impacts of revised offers in an annuities market in Chile.



We develop and estimate a structural two-stage model of firm competition and consumer choice. Firms send simultaneous initial offers, consumers decide whether to solicit revisions, and firms respond with revised offers; we recover preference and cost parameters from observed offer sequences and purchase decisions. Counterfactuals \footnote{\textcolor{red}{[what counterfactuals should we study?]}} show that permitting revised offers [IN(DE)CREASES] purchase rates and [INCREASES-lowers] average final prices, [RAISING/LOWERING]  aggregate consumer surplus, [MENTION DISTRIBUTIONAL IMPACTS].

\end{abstract}

\sepline
 
  
 

%%%%  \textbf{The effects of revised offers are ambiguous-> give an economic intuition. }

The effects of... , are ambiguous. On one hand,    ... .  On the other hand, ... if . 


There are two economic forces at play. First, ... . Second,.... . 

Therefore the object of interest are the joint distribution of search cost and price disutility and the joint distribution of search cost and private type. 

 
%%%%% \textbf{Our setting}


%%%% \textbf{What we do}

We study the impact of... .
The setting offers ... 

we leverage variation in the ... 

 

We first provide descriptive evidence of ... . In our data .... . 
%Revised offers are significantly higher than initial offers.  When asking a revised offers the improvement over the initial offer is on average \textbf{1.8} monthly wages. 

To study the welfare effects ... , we build a two‑stage model of firm competition and consumer choice. In a first stage, firms simultaneously send initial offers to consumers. Consumers then decide whether to request revised offers from firms, choose one of the initial offers, or not purchase. In the second stage, firms simultaneously send revised offers to consumers that requested them. Finally, consumers choose one of the revised offers, one of the initial offers, or not purchase. 

    We find that [RESULTS TO BE ADDED]

\vspace{.5cm}
    
%Review of the literature
    
Our paper contributes to three strands of the literature. First, we contribute to the  work on [WRITE HERE]

Second, we contribute to the empirical literature on
    
Finally, our work speaks to the literature on

\vspace{.5cm}
%Paper organization
    
The remainder of the paper is organized as follows. Section 2 describes our setting and  data. Section 3 provides descriptive evidence [WHAT DESCRIPTIVES]. Section  4 describes our model of competition with revised offers. Section 5 discusses the identification  and estimation of the model and the main results from the estimates. Section 6 discusses  our counterfactual analysis of the impacts of banning revised offers. Finally, Section 7 concludes.
    

\end{document}
