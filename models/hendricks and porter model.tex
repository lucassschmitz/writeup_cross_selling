\documentclass[12pt]{article}
%%%%%%%%%%%%%%%%%%%%%%%%%%%%%%%%%%%%%%%%%%%%%%%%%%%%%%%%%%%%%%%%%%%%%%%%%%%%%%%%%%%%%%%%%%%%%%%%%%%%%%%%%%%%%%%%%%%%%%%%%%%%%%%%%%%%%%%%%%%%%%%%%%%%%%%%%%%%%%%%%%%%%%%%%%%%%%%%%%%%%%%%%%%%%%%%%%%%%%%%%%%%%%%%%%%%%%%%%%%%%%%%%%%%%%%%%%%%%%%%%%%%%%%%%%%%
\usepackage{amsfonts}
\usepackage{eurosym}
\usepackage{geometry}
\usepackage{amsmath,amsthm,amssymb}
\usepackage{ulem} 
\usepackage{graphicx}
\usepackage{comment}
%\usepackage[sort,comma]{natbib}
\usepackage[utf8]{inputenc}
\usepackage{setspace}
\usepackage[backend=biber, style = apa]{biblatex}
\usepackage{placeins} % to separate sections

\usepackage{adjustbox}
\usepackage{array}
\usepackage{multirow}
\usepackage{graphicx}
\usepackage{subcaption}
\usepackage{pifont}
\usepackage{amssymb}
\usepackage{comment}
\usepackage[hang, flushmargin, bottom]{footmisc}
\usepackage{footnotebackref}
\usepackage{xcolor}
\usepackage{hyperref}
\usepackage{booktabs}
\usepackage{pifont}
\usepackage{caption}
\usepackage{float}
\usepackage{todonotes}
\setcounter{MaxMatrixCols}{10}


%\setlength{\bibsep}{0.3pt}
\setlength{\textfloatsep}{5pt}
\hypersetup{breaklinks=true,hypertexnames=false,colorlinks=true,citecolor = teal}
\captionsetup{font=normalsize}
\newcommand{\cmark}{\ding{51}}
\def\sym#1{\ifmmode^{#1}\else\(^{#1}\)\fi}
\renewcommand{\thetable}{\Roman{table}}
\geometry{verbose,tmargin=.9in,bmargin=1in,lmargin=.8in,rmargin=.8in,nomarginpar}
\makeatletter
\DeclareTextSymbolDefault{\textquotedbl}{T1}
\theoremstyle{plain}
\newtheorem{thm}{\protect\theoremname}
\theoremstyle{plain}
\newtheorem{prop}[thm]{\protect\propositionname}
\theoremstyle{definition}  % Add this line
\newtheorem{definition}[thm]{Definition}  % Add this line
\theoremstyle{remark}  % Add this line
\newtheorem{remark}[thm]{Remark}  % Add this line
\providecommand{\propositionname}{Proposition}
\providecommand{\theoremname}{Theorem}
\makeatother
\newtheorem{ass}[thm]{Assumption}
% \input{tcilatex}
\usepackage{tikz}
\usetikzlibrary{shapes.geometric, arrows, positioning}


\addbibresource{../references.bib}
\begin{document}


There is one neighbor firm and an arbitrary number of non-neighbor firms.
Let $X$ and $Z$ denote, respectively, the private and public signals on $V$,
the unknown value of the representative drainage tract.
The neighbor firm observes the realizations of $X$ and $Z$ prior to bidding
on the tract, while non-neighbor firms observe only the realization of $Z$.
Realizations of the random variables will be denoted by lowercase letters.
In what follows, we treat $z$ as given, and are explicit about the dependence
of the distributions on its value.
However, for notational convenience, we will suppress the dependence of
bidding strategies on $z$.

The essential feature of our model is that the information revealed by on-site
drilling of an adjacent tract by a neighbor firm is a sufficient statistic
for the information non-neighbor firms acquire from seismic surveys.
The assumption that this information is known to the neighbor firm is made
in order to obtain a precise characterization of the equilibrium and its
properties.
A more realistic, but less tractable, assumption is that the non-neighbor
firms have noisy, but private, estimates of tract value.
However, as long as the estimates of the non-neighbor firms are not too
informative, we can use the result by Milgrom and Weber (1985) on the upper
hemicontinuity of the equilibrium correspondence to argue that the behavioral
implications of this descriptively more accurate model are approximately the
same as those of a model in which the estimates of non-neighbor firms are
based on public information.

The strategy of non-neighbor firm $i$ is a distribution function $G_i(\cdot)$
over the non-negative real numbers.
Adopting the approach of Engelbrecht-Wiggans, Milgrom, and Weber, we summarize
the information of the neighbor firm by the real-valued random variable
$H = E[V \mid X, z]$.
We shall assume that $H$ has an atomless distribution $F(\cdot \mid z)$,
with finite mean $\bar{H}$.
The strategy of the neighbor firm can then be defined as a function $a$
which maps realizations of $H$, which are associated with the realizations
of $X$, into the non-negative real numbers.
We shall assume that $a(h)$ is a differentiable, strictly increasing function
on the range $(R, \infty)$, where $R$ is the reservation price, and denote its
inverse function on this interval by $T(b)$.

Define $G(b) = G_1(b)\cdots G_n(b)$ to be the distribution function of the
maximum of the bids submitted by the uninformed firms on the tract.
Given the strategy combination $(a, G_1, \ldots, G_n)$, the payoff to the
neighbor firm, when its estimate of $V$ is $h$, is the product of the
probability that its bid is highest and its expected value of the tract less
its bid:
\begin{equation}
G(a(h))\bigl(h - a(h)\bigr).
\end{equation}

If the drainage tract contains any oil, it is usually part of a pool which
the neighbor firm has discovered on the adjacent tract.
This makes the value of the drainage tract to each firm contingent upon the
manner in which production is allocated among the firms.
If the firms bargain to an efficient allocation, tract valuations are
identical across firms.
In many instances, however, competition leads to some dissipation of rents.
In these cases, the neighbor firm is likely to have a higher tract valuation
than the non-neighbor firm, since it can take the externality into account
and internalize its effects.

We parameterize the possible difference in tract valuations by letting the
expected value of the drainage tract to the non-neighbor firm be equal to
$E[H \mid z] - c$, where $c$ is a fixed, non-negative constant.
The expected payoff to non-neighbor firm $i$ which submits a bid $b$ greater
than $R$ is
\begin{equation}
E\!\left[H - b - c \mid b > a(h); z\right]
F(T(b)\mid z)\prod_{j \neq i} G_j(b).
\end{equation}

The first term in equation (2) is expected profits, conditional on winning
the tract (and hence $b > a(h)$).
The remaining terms represent the probabilities of outbidding the neighboring
firm and the other non-neighbors.
Ties at the reservation price are assumed to be settled by randomization.

A Bayesian Nash equilibrium for the bidding game is an $(n+1)$-tuple of
strategies $(a^*, G_1^*, \ldots, G_n^*)$ such that the expected payoff to each
firm conditional on its information is maximized, given the strategies
employed by the other firms.

\newpage 

We turn next to a characterization of the equilibrium bid distributions.
Define
\begin{equation}
\phi(h)
=
\exp\!\left(
- \int_h^{\infty}
\frac{f(s \mid z)\int_h^{s} F(u \mid z)\,du}
{cF(s \mid z)^2 + F(s \mid z)\int_h^{s} F(u \mid z)\,du}
\, ds
\right).
\end{equation}
Note that if $c$ is equal to zero, then $\phi(h)$ is equal to $F(h \mid z)$.
Our theorem is a restatement of the theorem proved by
Engelbrecht-Wiggans, Milgrom, and Weber, extended to auctions with
asymmetric tract valuations.
The proof is essentially the same as the one given by
Engelbrecht-Wiggans, Milgrom, and Weber, and is given in the Appendix.

\begin{thm}
The $(n+1)$-tuple $(a^{*},G_1^{*},\ldots,G_n^{*})$ is an equilibrium point
if and only if
\begin{equation}
G^{*}(b) =
\begin{cases}
1, & b > H - c, \\
\phi(T(b)), & R < b < H - c, \\
\phi(R), & 0 < b < R,
\end{cases}
\end{equation}
and
\begin{equation}
a^{*}(h) =
\begin{cases}
E[H \mid H < h; z] - c, & h > \bar{h}, \\
R, & \bar{h} \ge h \ge R, \\
0, & h < R,
\end{cases}
\end{equation}
where $\bar{h}$ solves
\[
E[H \mid H < \bar{h}; z] - c = R .
\]
\end{thm}

The theorem states that the supports of the equilibrium distribution
functions are identical, and consist of $\{0\}$ and the interval
$[R, H-c]$.
We interpret a zero bid as no bid.
The equilibrium strategy of the neighbor firm on $(R, H-c)$ is uniquely
determined by the condition that, in equilibrium, non-neighbor firms
must earn zero profits.
That is, suppose a non-neighbor firm submits an equilibrium bid $b$.
Then, since $a^{*}$ is strictly increasing at $b$, there is a unique
$h'$ such that $b = a^{*}(h')$.
The expected profit of the tract to the non-neighbor firm conditional
on the event that it wins is $E[H \mid H < h'; z] - b - c$.
Setting this equation equal to zero implies that
\[
a^{*}(h') = E[H \mid H < h'; z] - c.
\]

The equilibrium strategies of non-neighbor firms are indeterminate.
However, the equilibrium distribution function of the maximum
uninformed bid is unique.
It is chosen in order to induce the neighbor firm to bid according to
the function given above.
The two distributions differ in the probability of the events of no
bid, and of a reservation bid $R$.
$G^{*}$ possesses a mass point equal to $F(\bar{h})$ at $\{0\}$, and is
constant at this value on the interval $(0,R]$.
The distribution of the neighbor bid also possesses a mass point at
$\{0\}$, but it is equal to $F(R)$, which is less than $F(\bar{h})$.
The distribution is constant at $F(R)$ on the open interval $(0,R)$,
and then jumps discontinuously upward at $R$.
The value of the mass point at $R$ is equal to $F(\bar{h}) - F(R)$.
If $c$ is equal to zero, $G^{*}$ is identical to the distribution of the
informed bid on $(R,H)$.

The randomized strategies of non-neighbor firms are a direct consequence
of the assumption that the neighbor firm knows their estimates.
If non-neighbor firms bid according to a pure (and hence predictable)
strategy which specifies a bid for each realization of the public
information variables, the optimal response of the neighbor firm is to
bid slightly above the maximum non-neighbor bid if the tract is worth
more than this number, and not bid otherwise.
But this implies that on average the non-neighbor firm is certain to
lose, since it will win only those tracts whose expected value is less
than its bids.
By randomizing, non-neighbor firms can induce the neighbor firm to bid
according to a strategy in which it will lose profitable tracts some of
the time.
As a result, non-neighbor firms will earn positive expected profits on
some tracts, and it is only on average that their expected profits are
zero.













\end{document}