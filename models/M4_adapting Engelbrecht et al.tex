\documentclass[12pt]{article}
%%%%%%%%%%%%%%%%%%%%%%%%%%%%%%%%%%%%%%%%%%%%%%%%%%%%%%%%%%%%%%%%%%%%%%%%%%%%%%%%%%%%%%%%%%%%%%%%%%%%%%%%%%%%%%%%%%%%%%%%%%%%%%%%%%%%%%%%%%%%%%%%%%%%%%%%%%%%%%%%%%%%%%%%%%%%%%%%%%%%%%%%%%%%%%%%%%%%%%%%%%%%%%%%%%%%%%%%%%%%%%%%%%%%%%%%%%%%%%%%%%%%%%%%%%%%
\usepackage{amsfonts}
\usepackage{eurosym}
\usepackage{geometry}
\usepackage{amsmath,amsthm,amssymb}
\usepackage{ulem} 
\usepackage{graphicx}
\usepackage{comment}
%\usepackage[sort,comma]{natbib}
\usepackage[utf8]{inputenc}
\usepackage{setspace}
\usepackage[backend=biber, style = apa]{biblatex}
\usepackage{placeins} % to separate sections

\usepackage{adjustbox}
\usepackage{array}
\usepackage{multirow}
\usepackage{graphicx}
\usepackage{subcaption}
\usepackage{pifont}
\usepackage{amssymb}
\usepackage{comment}
\usepackage[hang, flushmargin, bottom]{footmisc}
\usepackage{footnotebackref}
\usepackage{xcolor}
\usepackage{hyperref}
\usepackage{booktabs}
\usepackage{pifont}
\usepackage{caption}
\usepackage{float}
\usepackage{todonotes}
\setcounter{MaxMatrixCols}{10}


%\setlength{\bibsep}{0.3pt}
\setlength{\textfloatsep}{5pt}
\hypersetup{breaklinks=true,hypertexnames=false,colorlinks=true,citecolor = teal}
\captionsetup{font=normalsize}
\newcommand{\cmark}{\ding{51}}
\def\sym#1{\ifmmode^{#1}\else\(^{#1}\)\fi}
\renewcommand{\thetable}{\Roman{table}}
\geometry{verbose,tmargin=.9in,bmargin=1in,lmargin=.8in,rmargin=.8in,nomarginpar}
\makeatletter
\DeclareTextSymbolDefault{\textquotedbl}{T1}
\theoremstyle{plain}
\newtheorem{thm}{\protect\theoremname}
\theoremstyle{plain}
\newtheorem{prop}[thm]{\protect\propositionname}
\theoremstyle{definition}  % Add this line
\newtheorem{definition}[thm]{Definition}  % Add this line
\theoremstyle{remark}  % Add this line
\newtheorem{remark}[thm]{Remark}  % Add this line
\providecommand{\propositionname}{Proposition}
\newtheorem{proposition}{Proposition}

\providecommand{\theoremname}{Theorem}
\makeatother
\newtheorem{ass}[thm]{Assumption}
% \input{tcilatex}
\usepackage{tikz}
\usetikzlibrary{shapes.geometric, arrows, positioning}


\addbibresource{../references.bib}
\begin{document}

I think this model is better than the model of Sharpe (model 2) because it is more general, for example the consumer type is allowed to have a generic distribution whereas in the case of Sharpe's model there are only two type of consumers. Moreover the equilibrium proof is easier and there is more litereature on this types of model(\cite{engelbrecht-wiggans_competitive_1983,hendricks_empirical_1988}), hence it would be easier to present and readers probably would prefer it over the Sharpe's model.




\section{Model}

Adapting \textcite{engelbrecht-wiggans_competitive_1983}

This version incorporates switching costs. Let bank 1 be the incumbent, which has a relationship with the borrower. The borrower incurs a switching cost $\lambda > 0$ if they choose bank 2. 
This borrower chooses bank 2 if  $r_1 > r_2 + \lambda$, otherwise it chooses bank 1. 

Define $h$ to be default probability, there is an informed bank (bank 1) and an uninformed bank (bank 2), which is smoothly distributed in the population according to the cdf $F(h)$ and pdf $f(h)$.   The strategies are $r_1(h) = \sigma(h): h\rightarrow r$ and $G(x) = \Pr (r_2\leq x)$, which are interest rates. 

Assume that  in equilibrium $\sigma$ is an increasing function, and denote by $\tau: r\rightarrow h $ its inverse, $\tau(\sigma(h)) = h$. 

Then expected profits of bank 1 are: 

\begin{align}
    \pi_1(r_1(h)) &= \Pr(\sigma(h) \leq r_2 + \lambda)\cdot [(1-h)\sigma(h)-1] \notag \\
     &= \Pr(r_2 \geq \sigma(h) - \lambda)\cdot [(1-h)\sigma(h)-1] \notag \\
     &=  [1-G(\sigma(h)-\lambda)]\cdot [(1-h)\sigma(h)-1] \label{eq:profit1}
\end{align}
and the expected profits of bank 2 are: 

Bank 2 wins if $r_2 < \sigma(h) - \lambda$, which is $\sigma(h) > r_2 + \lambda$.
\begin{align}
    \pi_2(r_2) &= \Pr(\sigma(h) > r_2 + \lambda) \cdot E[(1-h)\cdot r_2 -1 \mid \sigma(h) > r_2 + \lambda] \notag \\ 
    &= \Pr(h > \tau(r_2+\lambda)) \cdot E[(1-h)\cdot r_2 -1 \mid h > \tau(r_2+\lambda)] \notag \\ 
     &= [1-F(\tau(r_2+\lambda))] \cdot \left[E[(1-h) \mid h > \tau(r_2+\lambda)]\cdot r_2 -1 \right]
\end{align}

Assume that bank 2 makes zero profits\footnote{Would have to be proved, but I am confident that it is true since the uninformed firm in auction models always makes zero profits. }, then we have: 

\begin{align}
   [1-F(\tau(r_2+\lambda))] \cdot \left[E[(1-h) \mid h > \tau(r_2+\lambda)]\cdot r_2 -1 \right] =0
\end{align}
since the winning probability is not zero, then the expected profits have to be zero. 
\begin{align}\label{eq:profits2}
    E[(1-h) \mid h > \tau(r_2+\lambda)]\cdot r_2 -1  =0 \implies  r_2  =\frac{1}{E[(1-h) \mid h > \tau(r_2+\lambda)]} 
\end{align}


Let $k = \tau(r_2 + \lambda)$, which means $\sigma(k) = r_2 + \lambda$. Therefore $r_2 = \sigma(k) - \lambda$. Substituting:
\begin{align*}
    \sigma(k) - \lambda = \frac{1}{E[1-h | h > k]}
\end{align*}
Hence, 
\begin{align}\label{eq:strategy_home}
    \sigma(h)  = \lambda + \mu(h)^{-1} 
\end{align}
where $\mu(h) = E[1-H | H > h]$.

Then we can use profit maximization by the first firm, the FOC of equation \ref{eq:profit1} are: 
\begin{align}
    - g(\sigma(h)-\lambda) [ (1-h)\sigma(h)-1] + [1-G(\sigma(h)-\lambda)] [1-h] &= 0 \notag \\ 
    \frac{1-h}{ [ (1-h)\sigma(h)-1] } = \frac{g(\sigma(h)-\lambda)}{[1-G(\sigma(h)-\lambda)]} &= - \frac{d}{d\sigma} [\log(1-G(\sigma(h)-\lambda))] \notag \\
    \frac{1-\tau(r)}{ [ (1-\tau(r))r-1] }  = - \frac{d}{d\sigma} [\log(1-g(r-\lambda))]  
\end{align}
Integrating both sides from $\underline r = \sigma(\underline h)$ to a given $r$, where $G( \underline r - \lambda)=0 $ we have: 
\begin{align}
    -[
\log(1-G(r-\lambda)) - \log(1-\underbrace{G(\underline r - \lambda))}_{=0} ] = \int_{\underline r}^r \frac{1-\tau(u)}{ [ (1-\tau(u))u-1] } du \notag \\ 
     -\log(1-G(r-\lambda))  = \int_{\underline r}^r \frac{1-\tau(u)}{ [ (1-\tau(u))u-1] } du \notag \\ 
     G(r-\lambda) =1-\exp\left[ -\int_{\underline r}^r \frac{1-\tau(u)}{ [ (1-\tau(u))u-1] } du \right] \label{eq:mixed_strategy}
\end{align}

\subsection{Case with $\lambda = 0$}

From equation \ref{eq:strategy_home} we have that: 
\begin{align}    
    \sigma(h) =\frac{1}{E[(1-H) \mid H  \leq h]}   \equiv \frac{1}{\mu(h)}
\end{align}
Then we can obtain the mixed strategy of bank 2 by a change of variables, consider $u = \sigma(t) \implies \tau(u) = t, du = \sigma'(t)dt$, then the limits of integration change from $[\underline r, r]$ to $\underline h, \tau(r)]$. Substituting into equation \ref{eq:mixed_strategy} we have: 
\begin{align}
    G(\sigma(h)) = 1-\exp\left[ -\int_{\underline h}^h \frac{1-t}{ [ (1-t)\sigma(t)-1] }\sigma'(t) dt  \right] 
\end{align}
given that $\sigma(t) = 1/\mu(t)$, we have $\sigma'(t) = -\mu'(t) /\mu(t)^2 $, replacing in the equation above:
\begin{align}
    G(\sigma(h)) = 1-\exp\left[ \int_{\underline h}^h \frac{1-t}{\frac{1-t}{\mu(t)}-1 }\frac{\mu'(t)}{\mu(t)^2} dt  \right] = 1-\exp\left[ \int_{\underline h}^h \frac{(1-t)\mu'(t)}{(1-t-\mu(t))\mu(t)} dt  \right] 
\end{align}


\subsection{Example}

To derive a closed-form solution, we assume that the default risk $h$ is uniformly distributed on $[\underline{h},\bar{h}]$, where $0 < \underline{h} < \bar{h} < 1$. This ensures that even the safest borrower has positive default risk and that no borrower defaults with certainty. The conditional expectation is then:
\begin{align*}
    E[1-h \mid h > k] = \frac{\int_k^{\bar{h}} (1-x) dx}{\int_k^{\bar{h}} dx} = \frac{\left[(x - \frac{x^2}{2})\right]_k^{\bar{h}}}{\bar{h}-k} = \frac{(\bar{h} - \frac{\bar{h}^2}{2}) - (k - \frac{k^2}{2})}{\bar{h}-k}
\end{align*}
Simplifying the numerator:
\begin{align*}
    \bar{h} - \frac{\bar{h}^2}{2} - k + \frac{k^2}{2} = (\bar{h} - k) - \frac{\bar{h}^2 - k^2}{2} = (\bar{h} - k) - \frac{(\bar{h}-k)(\bar{h}+k)}{2} = (\bar{h}-k)\left(1 - \frac{\bar{h}+k}{2}\right)
\end{align*}
Therefore:
\begin{align*}
    E[1-h \mid h > k] = 1 - \frac{\bar{h}+k}{2} = \frac{2 - \bar{h} - k}{2}
\end{align*}
Substituting this into equation \ref{eq:profits2} with $k = \tau(r_2+\lambda)$:
\begin{align*}
    r_2 = \frac{1}{(2-\bar{h}-\tau(r_2+\lambda))/2} = \frac{2}{2-\bar{h}-\tau(r_2+\lambda)}
\end{align*}
We can solve this for $\tau$. Let $r=r_2+\lambda$, so $r_2=r-\lambda$. The equation becomes:
\begin{align*}
    r-\lambda = \frac{2}{2-\bar{h}-\tau(r)} \implies 2-\bar{h}-\tau(r) = \frac{2}{r-\lambda} \implies \tau(r) = 2 - \bar{h} - \frac{2}{r-\lambda}
\end{align*}
This is the inverse of bank 1's strategy. To find the strategy $\sigma(h)$ itself, we set $h = \tau(r)$ and solve for $r$:
\begin{align*}
    h = 2 - \bar{h} - \frac{2}{r-\lambda} \implies r-\lambda = \frac{2}{2-\bar{h}-h} \implies r = \sigma(h) = \lambda + \frac{2}{2-\bar{h}-h}
\end{align*}
This is the equilibrium pricing function for the informed bank. It prices at a markup over the switching cost $\lambda$, where the markup depends on the borrower's risk and the upper bound $\bar{h}$.

\textbf{Verification:} We can verify this makes economic sense:
\begin{itemize}
    \item For the safest borrower ($h = \underline{h}$): $\sigma(\underline{h}) = \lambda + \frac{2}{2-\bar{h}-\underline{h}}$
    \item For the riskiest borrower ($h = \bar{h}$): $\sigma(\bar{h}) = \lambda + \frac{2}{2-2\bar{h}} = \lambda + \frac{1}{1-\bar{h}}$
    \item Since $\bar{h} < 1$, the rate is finite and positive for all borrowers in the support.
    \item The profit margin for type $h$ is $(1-h)\sigma(h) - 1 = (1-h)\left(\lambda + \frac{2}{2-\bar{h}-h}\right) - 1$. 
\end{itemize}

Finally, we can use our expression for $\tau(u)$ to find the explicit distribution $G$ for bank 2 from equation \ref{eq:mixed_strategy}. We have:
\begin{align*}
    1 - \tau(u) = 1 - \left(2 - \bar{h} - \frac{2}{u-\lambda}\right) = \bar{h} - 1 + \frac{2}{u-\lambda}
\end{align*}
The integrand becomes:
\begin{align*}
    \frac{1-\tau(u)}{[ (1-\tau(u))u-1] } &= \frac{\bar{h} - 1 + \frac{2}{u-\lambda}}{\left(\bar{h} - 1 + \frac{2}{u-\lambda}\right)u - 1}
\end{align*}
Let $A = \bar{h} - 1$ (note $A < 0$ since $\bar{h} < 1$). Then:
\begin{align*}
    \frac{A + \frac{2}{u-\lambda}}{\left(A + \frac{2}{u-\lambda}\right)u - 1} &= \frac{A(u-\lambda) + 2}{(u-\lambda)} \cdot \frac{1}{\frac{[A(u-\lambda) + 2]u - (u-\lambda)}{u-\lambda}} \\
    &= \frac{A(u-\lambda) + 2}{Au^2 - A\lambda u + 2u - u + \lambda} \\
    &= \frac{A(u-\lambda) + 2}{Au^2 + (2-A\lambda-1)u + \lambda} \\
    &= \frac{Au - A\lambda + 2}{Au^2 + (1-A\lambda)u + \lambda}
\end{align*}
Substituting back $A = \bar{h} - 1$:

\begin{align*}
    &= \frac{(\bar{h}-1)u - (\bar{h}-1)\lambda + 2}{(\bar{h}-1)u^2 + (1-(\bar{h}-1)\lambda)u + \lambda} \\
    &= \frac{(\bar{h}-1)(u-\lambda) + 2}{(\bar{h}-1)u^2 + (1-(\bar h -1)\lambda)u + \lambda}
\end{align*}

This integral is complex in the general case.
For a cleaner closed-form solution, consider the case where $\bar{h} = 1$ (so default probability ranges from $\underline{h} > 0$ to $1$). With $\bar{h} = 1$:
\begin{align*}
    \sigma(h) = \lambda + \frac{2}{2-1-h} = \lambda + \frac{2}{1-h}
\end{align*}
and
\begin{align*}
    \tau(r) = 2 - 1 - \frac{2}{r-\lambda} = 1 - \frac{2}{r-\lambda}
\end{align*}
The lower bound of integration is $\underline{r} = \sigma(\underline{h}) = \lambda + \frac{2}{1-\underline{h}}$.

Now we compute:
\begin{align*}
    1 - \tau(u) = 1 - \left(1 - \frac{2}{u-\lambda}\right) = \frac{2}{u-\lambda}
\end{align*}
The integrand becomes:
\begin{align*}
    \frac{1-\tau(u)}{(1-\tau(u))u - 1} = \frac{\frac{2}{u-\lambda}}{\frac{2u}{u-\lambda} - 1} = \frac{\frac{2}{u-\lambda}}{\frac{2u - (u-\lambda)}{u-\lambda}} = \frac{2}{u + \lambda}
\end{align*}
The integral becomes:
\begin{align*}
    \int_{\underline{r}}^{r} \frac{2}{u+\lambda} du = 2\ln(u+\lambda)\Big|_{\underline{r}}^{r} = 2\ln\left(\frac{r+\lambda}{\underline{r}+\lambda}\right) = \ln\left(\left(\frac{r+\lambda}{\underline{r}+\lambda}\right)^2\right)
\end{align*}
Substituting into equation \ref{eq:mixed_strategy}:
\begin{align*}
    -\log(1-G(r-\lambda)) &= \ln\left(\left(\frac{r+\lambda}{\underline{r}+\lambda}\right)^2\right) \\
    \implies G(r-\lambda) &= 1 - \left(\frac{\underline{r}+\lambda}{r+\lambda}\right)^2
\end{align*}
Let $x = r - \lambda$, so $r = x + \lambda$. Bank 2's strategy $G(x) = \Pr(r_2 \le x)$ is:
\begin{align*}
    G(x) = 1 - \left(\frac{\underline{r}+\lambda}{x + 2\lambda}\right)^2
\end{align*}
where $\underline{r} + \lambda = 2\lambda + \frac{2}{1-\underline{h}}$.

The support for bank 2's offers starts where $G(x) = 0$, which occurs at $x = \underline{r} - \lambda = \frac{2}{1-\underline{h}}$.

\subsection{Switchers }

\textcolor{red}{What is the probability a bank switches?}

\newpage 

\subsection{Equilibrium computation}

Previosuly we derived the equilibrium conditions (see equations \ref{eq:strategy_home} and \ref{eq:mixed_strategy}) for the informed incumbent bank and the uninformed entrant bank.

For a generic distribution $F(h)$ with support $[h_{min}, h_{max}]$, we cannot obtain closed-form solutions. Instead, we use numerical methods to compute the equilibrium strategies.

\subsubsection{Step 1: Define the conditional expectation function}

For each type $h$, we need to compute the conditional expectation:
\begin{align}
    \mu(h) = E[1-H \mid H > h] = \frac{\int_h^{h_{max}} (1-t) f(t) dt}{1 - F(h)}
\end{align}

\textbf{Properties:}
\begin{itemize}
    \item $\mu(h)$ is decreasing in $h$ (higher types have worse expected repayment)
    \item $\mu(h_{min}) = E[1-H]$ (unconditional expectation)
    \item As $h \to h_{max}$: $\mu(h) \to 1 - h_{max}$
\end{itemize}

\textbf{Empirical computation:} Given $N$ draws from $F$: $\{h_1, ..., h_N\}$, we compute $\mu$ on a grid of $M$ points $\{h_1^{grid}, ..., h_M^{grid}\}$:
\begin{align}
    \mu(h_k^{grid}) = \frac{1}{|\{i: h_i > h_k^{grid}\}|} \sum_{i: h_i > h_k^{grid}} (1 - h_i)
\end{align}


\subsubsection{Step 2: Support of strategies}

The incumbent's offers range from:
\begin{align}
    \underline{r} &= \sigma(h_{min}) = \lambda + \frac{1}{E[1-H]} \\
    \bar{r} &= \sigma(h_{max}) = \lambda + \frac{1}{1-h_{max}}
\end{align}

The entrant's offers have support $[x_{min}, \infty)$ where $x_{min} = \underline{r} - \lambda = \frac{1}{E[1-H]}$.

\subsubsection{Step 3: Compute the inverse function $\tau(r)$}

From $\sigma(\tau(r)) = r$, we need to find $\tau(r)$ for each $r \in [\underline{r}, \bar{r}]$. Given $\sigma(h) = \lambda + 1/\mu(h)$:
\begin{align}
    r = \lambda + \frac{1}{\mu(\tau(r))} \implies \mu(\tau(r)) = \frac{1}{r - \lambda}
\end{align}

\textbf{Numerical solution:} For each $r$ in a grid, solve for $\tau(r)$ by finding $h$ such that $\mu(h) = 1/(r-\lambda)$ using interpolation or root-finding.

\subsubsection{Step 4: Compute the entrant's mixed strategy}

From equation \ref{eq:mixed_strategy}, we compute:
\begin{align}
    G(r-\lambda) = 1 - \exp\left[-\int_{\underline{r}}^r \frac{1-\tau(u)}{(1-\tau(u))u - 1} du\right]
\end{align}

\textbf{Numerical algorithm:}
\begin{enumerate}
    \item Create a grid of $M$ points: $r_1 = \underline{r}, r_2, ..., r_M$ in $[\underline{r}, \bar{r}]$
    \item For each $r_j$, compute $\tau(r_j)$ as described in Step 3
    \item Compute the integrand: $I(r_j) = \frac{1-\tau(r_j)}{(1-\tau(r_j))r_j - 1}$
    \item Use numerical integration (e.g., trapezoidal rule): 
    \begin{align}
        \int_{\underline{r}}^{r_j} I(u) du \approx \sum_{k=1}^{j-1} \frac{I(r_k) + I(r_{k+1})}{2} (r_{k+1} - r_k)
    \end{align}
    \item Compute: $G(r_j - \lambda) = 1 - \exp\left[-\int_{\underline{r}}^{r_j} I(u) du\right]$
\end{enumerate}

\subsubsection{Step 5: Sample from the entrant's strategy}

To simulate the model with $N$ borrowers:
\begin{enumerate}
    \item Draw $N$ borrowers: $h_i \sim F(h)$ for $i = 1, ..., N$
    \item For each borrower $i$, compute incumbent's offer: $r_1^i = \sigma(h_i) = \lambda + 1/\mu(h_i)$
    \item Draw entrant's offers from $G$ using inverse transform sampling:
    \begin{itemize}
        \item Draw $U_i \sim \text{Uniform}[0,1]$
        \item Find $r_2^i$ such that $G(r_2^i) = U_i$ by interpolation
    \end{itemize}
    \item Borrower $i$ switches if $r_2^i + \lambda < r_1^i$
\end{enumerate}

\subsubsection{Verification}

The numerical solution should satisfy:
\begin{itemize}
    \item Bank 2's expected profit is zero for all $r_2$ in its support
    \item Bank 1 is indifferent over its support for each type $h$
    \item $G$ is a valid CDF: $G(x_{min}) = 0$ and $\lim_{x \to \infty} G(x) = 1$
\end{itemize}

 


\end{document}

