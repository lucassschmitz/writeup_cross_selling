\documentclass[12pt]{article}
%%%%%%%%%%%%%%%%%%%%%%%%%%%%%%%%%%%%%%%%%%%%%%%%%%%%%%%%%%%%%%%%%%%%%%%%%%%%%%%%%%%%%%%%%%%%%%%%%%%%%%%%%%%%%%%%%%%%%%%%%%%%%%%%%%%%%%%%%%%%%%%%%%%%%%%%%%%%%%%%%%%%%%%%%%%%%%%%%%%%%%%%%%%%%%%%%%%%%%%%%%%%%%%%%%%%%%%%%%%%%%%%%%%%%%%%%%%%%%%%%%%%%%%%%%%%
\usepackage{amsfonts}
\usepackage{eurosym}
\usepackage{geometry}
\usepackage{amsmath,amsthm,amssymb}
\usepackage{ulem} 
\usepackage{graphicx}
\usepackage{comment}
%\usepackage[sort,comma]{natbib}
\usepackage[utf8]{inputenc}
\usepackage{setspace}
\usepackage[backend=biber, style = apa]{biblatex}
\usepackage{placeins} % to separate sections

\usepackage{adjustbox}
\usepackage{array}
\usepackage{multirow}
\usepackage{graphicx}
\usepackage{subcaption}
\usepackage{pifont}
\usepackage{amssymb}
\usepackage{comment}
\usepackage[hang, flushmargin, bottom]{footmisc}
\usepackage{footnotebackref}
\usepackage{xcolor}
\usepackage{hyperref}
\usepackage{booktabs}
\usepackage{pifont}
\usepackage{caption}
\usepackage{float}
\usepackage{todonotes}
\setcounter{MaxMatrixCols}{10}


%\setlength{\bibsep}{0.3pt}
\setlength{\textfloatsep}{5pt}
\hypersetup{breaklinks=true,hypertexnames=false,colorlinks=true,citecolor = teal}
\captionsetup{font=normalsize}
\newcommand{\cmark}{\ding{51}}
\def\sym#1{\ifmmode^{#1}\else\(^{#1}\)\fi}
\renewcommand{\thetable}{\Roman{table}}
\geometry{verbose,tmargin=.9in,bmargin=1in,lmargin=.8in,rmargin=.8in,nomarginpar}
\makeatletter
\DeclareTextSymbolDefault{\textquotedbl}{T1}
\theoremstyle{plain}
\newtheorem{thm}{\protect\theoremname}
\theoremstyle{plain}
\newtheorem{prop}[thm]{\protect\propositionname}
\theoremstyle{definition}  % Add this line
\newtheorem{definition}[thm]{Definition}  % Add this line
\theoremstyle{remark}  % Add this line
\newtheorem{remark}[thm]{Remark}  % Add this line
\providecommand{\propositionname}{Proposition}
\providecommand{\theoremname}{Theorem}
\makeatother
\newtheorem{ass}[thm]{Assumption}
% \input{tcilatex}
\usepackage{tikz}
\usetikzlibrary{shapes.geometric, arrows, positioning}


\addbibresource{references.bib}
\begin{document}

 

 

    \subsection{Model 1}
 
    Consider the simplest model of multi-product firms with switching costs, we assume that there are two periods and two products. For example one can think that initially a consumer opens a checking account (product 1) and later on she may take a loan (product 2). We denote the period/product by  $t = 1,2$. 

    There are $J$ firms, indexed by $j$. 

    \paragraph{Consumer problem}

    The per period utility of the consumer is given by: 

    \begin{equation}
        u_{ijt} = \beta_t - \alpha p_{ijt} + \xi_{ijt} + \epsilon_{ijt} 
    \end{equation}
    where $\beta_t$ is the valuation for the product, $p_{ijt}$ is the price and $\xi_{ijt}$ is a demand shock. 

    If the consumer buys from differnt firm in each period, she incurs a switching cost $s$. Denote by $j_t$ the firm chosen in period $t$, then the total utility across both periods is given by:

    \begin{equation}
        U_i(j_1, j_2) = u_{ij_11} + u_{ij_22} - s \cdot  \mathbb{I}(j_1 \neq j_2)
    \end{equation}
    and the optimal choice is given by: 
    \begin{equation}
        (j_1^*, j_2^*) = \arg \max_{j_1, j_2} U_i(j_1, j_2)
    \end{equation}

    %Denote by $D_{j1}(p_1) = \Pr(j_1^* = j; p_1)$ and $D_{j2}(p_1, p_2) = \Pr(j_2^* = j; p_1, p_2)$  the demand functions for period 1 and period 2 respectively.


    Denote by $D_{1j}(p_1) = \Pr(j_1^* = j; p_1)$ and $D_{2j}(j_1, p_2)  = \Pr(j_2^* = j; j_1, p_2)$  the demand functions for period 1 and period 2 respectively.

    Then:  

    \begin{equation}
        D_{2j}(j_1, p_2) =  
        \begin{cases}
            \frac{\exp (\delta_{ijt})}{\exp (\delta_{ijt})+ \sum_{j' \neq j}\exp (\delta_{ijt} -\alpha s)} &; j= j_1 \\
            \frac{\exp (\delta_{ijt} - \alpha s)}{\exp (\delta_{ij_1t})+ \sum_{j' \neq j_1}\exp (\delta_{ijt} -\alpha s)} &; j \neq j_1
        \end{cases}    
    \end{equation}

    Denote by $p_2(p_1)$ the vector of prices in period 2 given the prices in period 1 and by $S_2(p_2;j_1)$ the surplus function of period 2 given the prices and the firm chosen in period 1. Then in the firt period the consumer maximizes: 

    \begin{equation}
        \max_{j_1} u_{ij_11} +  S_2(p_2(p_1);j_1) 
    \end{equation}

    where $p_2(p_1)$ will be determined in equilibrium. 


    \paragraph{Firm problem}

    In the second period the firm chooses prices $p_{j2}(j_1)$ taking as given the firm chosen in period 1, to maximize profits:

    \begin{equation}
        \max_{p} D_{2j}(j_1, (p, p_{-j2}^*(j_1)))(p-c_2) 
    \end{equation}


    In the first period each firm chooses $p_{j1}$ according to:  

    \begin{equation}
        \max_{p_{j1}} D_{1j}((p_{j1}, p_{-j1}^*)) (p_{j1} - c_1) + \sum_{j_1} D_{2j}(j_1, (p_{2}^*(j_1))) (p_{j2}^*(j_1) - c_2) 
    \end{equation}

    This model is essentially the same as Dube et al. (2009), but in a two-period setting. 





 
\end{document}