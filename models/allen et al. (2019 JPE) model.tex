\documentclass[12pt]{article}
%%%%%%%%%%%%%%%%%%%%%%%%%%%%%%%%%%%%%%%%%%%%%%%%%%%%%%%%%%%%%%%%%%%%%%%%%%%%%%%%%%%%%%%%%%%%%%%%%%%%%%%%%%%%%%%%%%%%%%%%%%%%%%%%%%%%%%%%%%%%%%%%%%%%%%%%%%%%%%%%%%%%%%%%%%%%%%%%%%%%%%%%%%%%%%%%%%%%%%%%%%%%%%%%%%%%%%%%%%%%%%%%%%%%%%%%%%%%%%%%%%%%%%%%%%%%
\usepackage{amsfonts}
\usepackage{eurosym}
\usepackage{geometry}
\usepackage{amsmath,amsthm,amssymb}
\usepackage{ulem} 
\usepackage{graphicx}
\usepackage{comment}
%\usepackage[sort,comma]{natbib}
\usepackage[utf8]{inputenc}
\usepackage{setspace}
\usepackage[backend=biber, style = apa]{biblatex}
\usepackage{placeins} % to separate sections

\usepackage{adjustbox}
\usepackage{array}
\usepackage{multirow}
\usepackage{graphicx}
\usepackage{subcaption}
\usepackage{pifont}
\usepackage{amssymb}
\usepackage{comment}
\usepackage[hang, flushmargin, bottom]{footmisc}
\usepackage{footnotebackref}
\usepackage{xcolor}
\usepackage{hyperref}
\usepackage{booktabs}
\usepackage{pifont}
\usepackage{caption}
\usepackage{float}
\usepackage{todonotes}
\setcounter{MaxMatrixCols}{10}


%\setlength{\bibsep}{0.3pt}
\setlength{\textfloatsep}{5pt}
\hypersetup{breaklinks=true,hypertexnames=false,colorlinks=true,citecolor = teal}
\captionsetup{font=normalsize}
\newcommand{\cmark}{\ding{51}}
\def\sym#1{\ifmmode^{#1}\else\(^{#1}\)\fi}
\renewcommand{\thetable}{\Roman{table}}
\geometry{verbose,tmargin=.9in,bmargin=1in,lmargin=.8in,rmargin=.8in,nomarginpar}
\makeatletter
\DeclareTextSymbolDefault{\textquotedbl}{T1}
\theoremstyle{plain}
\newtheorem{thm}{\protect\theoremname}
\theoremstyle{plain}
\newtheorem{prop}[thm]{\protect\propositionname}
\theoremstyle{definition}  % Add this line
\newtheorem{definition}[thm]{Definition}  % Add this line
\theoremstyle{remark}  % Add this line
\newtheorem{remark}[thm]{Remark}  % Add this line
\providecommand{\propositionname}{Proposition}
\providecommand{\theoremname}{Theorem}
\makeatother
\newtheorem{ass}[thm]{Assumption}
% \input{tcilatex}
\usepackage{tikz}
\usetikzlibrary{shapes.geometric, arrows, positioning}


\addbibresource{../references.bib}
\begin{document}

\section{Model}
\subsection{Preferences and Cost Functions}

Consumers solve a discrete-choice problem over which lender to use to finance their mortgage:
\begin{equation}
\max_{j \in \mathcal{J}} v_j - p_j,
\end{equation}
where $\mathcal{J}$ is the set of lenders offering a quote, $p_j$ denotes the monthly payment offered by lender $j$, and $v_j$ denotes the maximum willingness to pay (WTP) associated with bank $j$.

The choice set $\mathcal{J}$ is defined both by where consumers live and by their search decision.
Consumers can obtain a quote from their home bank $(h)$ and from the $n$ lenders in $\mathcal{N}$.
We assume that the cost of obtaining a quote from the home bank is zero, while the cost of getting additional quotes is $k>0$.
This search cost does not depend on the number of quotes and is distributed in the population according to the cumulative distribution function (CDF) $H(\cdot)$.

The WTP of consumers is a combination of differentiation and mortgage valuation:
\begin{equation}
v_j =
\begin{cases}
\bar{v} + \lambda & \text{if } j = h, \\
\bar{v} & \text{else.}
\end{cases}
\end{equation}

The valuation for a mortgage, $\bar{v}$, is common across all lenders.
Throughout we assume that it is large enough not to affect the set of consumers present in our sample.
The parameter $\lambda \ge 0$ measures consumers' WTP for their home bank relative to other lenders.

We also assume that banks have a constant borrower-specific marginal cost of lending.
This measures the direct lending costs for the bank (i.e., default and prepayment risks), net of the future benefits associated with selling complementary services to the borrower.
Since we do not observe the performance of the contract along the risk and complementarity dimensions, we use a reduced-form function to approximate the net present value of the contract.
The monthly cost for bank $j$ to lend to the consumer is
\begin{equation}
c_j =
\begin{cases}
c - \Delta & \text{if } j = h, \\
c + \omega_j & \text{if } j \ne h,
\end{cases}
\end{equation}
where $c$ is the common cost of lending to the consumer; $\omega_j$ is the cost differential of lender $j$ relative to the home bank (or its match value); and $\Delta$ is the home bank’s cost advantage.

This advantage arises because of the multiproduct nature of financial institutions and the fact that the home bank is potentially already selling profitable products to the consumer.
It could come from real complementarities generated by bundling products (economies of scope) or from the fact that costs include not just the direct cost of mortgage lending but also revenues/costs derived from the sale of additional products.
In contrast to the home bank, competing lenders may need to offer discounts on these products to overcome the switching costs or may not earn any revenues at all from them if consumers do not switch.

As we will see below, the importance of brand loyalty in the market is driven by the sum of the cost and WTP advantages of the home bank:
\[
\gamma = \Delta + \lambda.
\]
We refer to $\gamma$ as the home bank loyalty advantage.

The idiosyncratic component, $\omega_j$, is distributed according to $G(\cdot)$, with $E(\omega_j)=0$.
We use subscript $(k)$ to denote the $k$th-lowest cost match value among the non--home bank lenders.
The CDF of the $k$th-order statistic among the $n$ lenders is given by
\[
G_{(k)}(w \mid n) = \Pr(\omega_{(k)} < w \mid n).
\]
\newpage 


Finally, lenders’ quotes are constrained by a common posted price $\bar{p}$.
The posted price determines both the reservation price of consumers
(i.e., $\bar{v} > \bar{p}$) and whether or not consumers qualify for a loan at a given
lender (i.e., $\bar{p} > c_j$).

\subsection{Bargaining Protocol, Information, and Timing of the Game}

In an initial period outside the model, consumers choose the type of house
they want to buy, the loan size, $L$, and the timing of the home purchase
(including closing date).
Our focus is on the negotiation process, which we model as a two-stage game.
In the first stage, the home bank makes an initial offer $p^0$.
At this point, the borrower can accept the offer or search for additional
quotes by paying the search cost $k$.
If the initial quote is rejected, the borrower organizes an English auction
among the home bank and the $n$ other banks present in his or her neighborhood.
The lender choice maximizes the utility of consumers, as in equation (1).

Information about costs and preferences is revealed sequentially.
At the initial stage, all parties observe the posted price $\bar{p}$, the number
of rival banks $n$, the common component of the lending cost $c$, and the
home bank cost and WTP advantages $(\lambda,\Delta)$.
These variables define the observed state vector:
\[
s = (c,\lambda,\Delta,\bar{p},n).
\]
This information is common to all players.
The search cost is privately observed by consumers.
The home bank knows only the distribution, which can vary across consumers
on the basis of observed demographic attributes.
Finally, in the second stage of the game, each lender learns its idiosyncratic
lending cost, $\omega_j$.

Before solving the game, two remarks are in order.
First, consumers are price takers in the model, and so lenders have full
bargaining power.
This does not mean, however, that consumers have no bargaining leverage,
since they have an informational advantage from knowing their search cost.
This prevents the home bank from extracting the entire surplus of consumers,
as in Allen et al.\ (2014a).
Second, consumers are assumed

\newpage 

Before solving the game, two remarks are in order.
First, consumers are price takers in the model, and so lenders have full
bargaining power.
This does not mean, however, that consumers have no bargaining leverage,
since they have an informational advantage from knowing their search cost.
This prevents the home bank from extracting the entire surplus of consumers,
as in Allen et al.\ (2014a).
Second, consumers are assumed to pay the cost of generating offers at the
auction stage (rather than firms).
Therefore, banks that are not competitive relative to the home bank are,
in theory, indifferent between submitting and not submitting a quote.
In these cases we assume that banks always submit a truthful offer that is
consistent with their realized match values.

Next, we describe the solution of the negotiation by backward induction,
starting with the competition stage.

\subsection{Competition Stage}

Conditional on rejecting $p^0$, the home bank competes with lenders in the
borrower’s choice set.
We model competition as an English auction with heterogeneous firms and a
cost advantage for the home bank.
Since the initial quote can be recalled, firms face a reservation price:
$p^0 \le \bar{p}$.

We can distinguish between two cases leading to a transaction:
(i) $\bar{p} < c - \Delta$ and (ii) $c < p^0 + \Delta \le \bar{p} + \Delta$.
In the first case the borrower does not qualify at the home bank.
Borrowers not qualifying at their home bank must search, and their
reservation price is $\bar{p}$.
This borrower may qualify at other banks because of differences in $\omega_j$.
The lowest-cost qualifying bank wins by offering a price equal to the lending
cost of the second most efficient qualifying lender:
\begin{equation}
p^{*} = \min\{c + \omega_{(2)}, \bar{p}\}.
\end{equation}
This occurs if and only if $0 < \bar{p} - c - \omega_{(1)}$.

If the borrower qualifies at the home bank, the highest-surplus bank wins and
offers a quote that provides the same utility as the second-best option.
The equilibrium pricing function is
\begin{equation}
p^{*} =
\begin{cases}
p^0
& \text{if } \bar{v} + \lambda - p^0 \ge \bar{v} - c - \omega_{(1)}, \\[4pt]
c + \omega_{(1)} + \lambda
& \text{if } \bar{v} + \lambda - p^0 < \bar{v} - c - \omega_{(1)} < \bar{v} - c + \gamma, \\[4pt]
c - \gamma
& \text{if } \bar{v} - c - \omega_{(1)} > \bar{v} - c + \gamma > \bar{v} - c - \omega_{(2)}, \\[4pt]
c + \omega_{(2)}
& \text{if } \bar{v} - c - \omega_{(2)} > \bar{v} - c + \gamma.
\end{cases}
\end{equation}

This equation highlights the fact that, at the competition stage, lenders
directly competing with the home bank will, on average, have to offer a
discount equal to the loyalty advantage in order to attract new customers.
In cases 1 and 2 the home bank provides the highest utility and so wins the
auction.
In case 1, the initial quote provides higher utility than does the next-best
lender’s quote, and so the consumer pays $p^0$.
In case 2, the reverse is true, and so the consumer pays $c + \omega_{(1)} + \lambda$
and gets utility of $\bar{v} - c - \omega_{(1)}$.
In cases 3 and 4, the home bank is not the highest-surplus lender, and the
consumer pays $c - \gamma$ or $c + \omega_{(2)}$, depending on whether the
home bank is the second- or third-highest surplus lender.


\newpage 


\subsection{Search Decision and Initial Quote}

The borrower chooses to search by weighing the value of accepting $p^0$
or paying a sunk cost $\kappa$ to search in order to lower his expected
monthly payment.
The utility gain from search is
\begin{align}
\bar{\kappa}(p^0,s)
&= \bar{v} + \lambda\bigl[1 - G_{(1)}(-\gamma)\bigr]
- E[p^{*}\mid p^0,s]
- \bigl[\bar{v} + \lambda - p^0\bigr] \notag \\
&= p^0 - E[p^{*}\mid p^0,s] - \lambda G_{(1)}(-\gamma),
\end{align}
where $1 - G_{(1)}(-\gamma)$ is the retention probability of the home bank
in the competition stage.
A consumer will reject $p^0$ if and only if the gain from search is larger
than the search cost.
Therefore, the search probability is
\begin{equation}
\Pr\!\left(\kappa < p^0 - E[p^{*}\mid p^0,s] - \lambda G_{(1)}(-\gamma\mid n)\right)
\equiv H\!\left(\bar{\kappa}(p^0,s)\right).
\end{equation}

Lenders do not commit to a fixed interest rate and are open to haggling
with consumers on the basis of their outside options.
This allows the home bank to discriminate by offering the same consumer
up to two quotes:
(i) an initial quote $p^0$ and (ii) a competitive quote $p^{*}$ if the first
is rejected.

The price discrimination problem is based on the expected value of shopping
and the distribution of search costs.
More specifically, anticipating the second-stage outcome, the home bank
chooses $p^0$ to maximize its expected profit:
\begin{equation}
\max_{p^0 \le \bar{p}}
\ (p^0 - c + \Delta)\bigl[1 - H(\bar{\kappa}(p^0,s))\bigr]
+ H(\bar{\kappa}(p^0,s))\,E(\pi_h^{*}\mid p^0,s),
\end{equation}
where
\begin{align}
E(\pi_h^{*}\mid p^0,s)
&= (p^0 - c + \Delta)\bigl[1 - G_{(1)}(p^0 - \lambda - c)\bigr]
+ \int_{-\gamma}^{\,p^0 - c - \lambda} \bigl[\omega_{(1)} + \gamma\bigr]\,
dG_{(1)}
\end{align}
are the expected profits from the auction for the home bank.
The first term represents the case in which the initial quote provides
higher utility than the next-highest surplus lender, while the second is
the reverse.

Importantly, the home bank will offer a quote only if it makes positive
profit at the posted price:
$0 < \bar{p} - c + \Delta$.
In the interior solution, the optimal initial quote is implicitly defined
by the following first-order condition:
\begin{align}
p^0 - c + \Delta
&=
\underbrace{
\frac{1 - H(\bar{\kappa}(p^0,s))}
{H'(\bar{\kappa}(p^0,s))\,\bar{\kappa}_{p^0}(p^0,s)}
}_{\text{Search cost distribution}}
+
\underbrace{
E(\pi_h^{*}\mid p^0,s)
}_{\text{Cost and quality differentiation}} \notag \\
&\quad +
\underbrace{
\frac{H(\bar{\kappa}(p^0,s))}
{H'(\bar{\kappa}(p^0,s))\,\bar{\kappa}_{p^0}(p^0,s)}
\frac{\partial E(\pi_h^{*}\mid p^0,s)}{\partial p^0}
}_{\text{Reserve price effect}} .
\end{align}

where
\[
\bar{\kappa}_{p^0}(p^0,s)
= \frac{\partial \bar{\kappa}(p^0,s)}{\partial p^0}.
\]


\newpage 

Equation (6) implicitly defines the home bank’s profit margins from price
discrimination.
It highlights three sources of profits:
(i) positive average search costs,
(ii) market power from differentiation in cost and quality (i.e., match value
differences and home bank cost advantage), and
(iii) the reserve price effect.
If firms are homogeneous, the only source of profits will stem from the
ability of the home bank to offer higher quotes to high-search cost consumers.

Although the initial quote does not have a closed-form solution, in the
following proposition (proven in app.\ B) we claim that, in the interior, it
is additive in the common cost shock.
This simplifies the problem, since we need to numerically solve the
first-order condition for only one value of $c$ per consumer.

\begin{prop}
The optimal initial quote, $p^0$, is additive in $c$ in the interior:
$p^0 = c + \mu(\Delta,\lambda,n)$.
\end{prop}

From this proposition, we can characterize the initial quote as follows:
\begin{equation}
p^0(s) =
\begin{cases}
\bar{p}
& \text{if } c > \bar{p} - \mu(\Delta,\lambda,n), \\
c + \mu(\Delta,\lambda,n)
& \text{else.}
\end{cases}
\end{equation}

To summarize, the model predicts three equilibrium functions:
(i) the initial quote $p^0(s)$,
(ii) the search cost threshold $\bar{\kappa}(s)$, and
(iii) the competitive price $p^{*}(\omega,s)$.
Although it is difficult to characterize these functions analytically, the
separability of the initial quote in the interior leads to a series of useful
predictions that are summarized in corollary 1.
We use these implications in the identification section below.

\begin{remark}
The following predictions about the distribution of prices and search
probability in the interior, when $p^0(s) < \bar{p}$, follow from
proposition 1:
\begin{enumerate}
\item The equilibrium search probability is independent of $c$.
\item The equilibrium search probability is affected symmetrically by
$\lambda$ and $\Delta$.
\item The distribution of $p^{*}$ for switchers is a function only of
$\gamma = \lambda + \Delta$.
\item The average transaction price paid by loyal consumers is affected
asymmetrically by $\lambda$ and $\Delta$, and the effect of $\lambda$ is
stronger.
\end{enumerate}
\end{remark}


\end{document}