\documentclass[12pt]{article}
%%%%%%%%%%%%%%%%%%%%%%%%%%%%%%%%%%%%%%%%%%%%%%%%%%%%%%%%%%%%%%%%%%%%%%%%%%%%%%%%%%%%%%%%%%%%%%%%%%%%%%%%%%%%%%%%%%%%%%%%%%%%%%%%%%%%%%%%%%%%%%%%%%%%%%%%%%%%%%%%%%%%%%%%%%%%%%%%%%%%%%%%%%%%%%%%%%%%%%%%%%%%%%%%%%%%%%%%%%%%%%%%%%%%%%%%%%%%%%%%%%%%%%%%%%%%
\usepackage{amsfonts}
\usepackage{eurosym}
\usepackage{geometry}
\usepackage{amsmath,amsthm,amssymb}
\usepackage{ulem} 
\usepackage{graphicx}
\usepackage{comment}
%\usepackage[sort,comma]{natbib}
\usepackage[utf8]{inputenc}
\usepackage{setspace}
\usepackage[backend=biber, style = apa]{biblatex}
\usepackage{placeins} % to separate sections

\usepackage{adjustbox}
\usepackage{array}
\usepackage{multirow}
\usepackage{graphicx}
\usepackage{subcaption}
\usepackage{pifont}
\usepackage{amssymb}
\usepackage{comment}
\usepackage[hang, flushmargin, bottom]{footmisc}
\usepackage{footnotebackref}
\usepackage{xcolor}
\usepackage{hyperref}
\usepackage{booktabs}
\usepackage{pifont}
\usepackage{caption}
\usepackage{float}
\usepackage{todonotes}
\setcounter{MaxMatrixCols}{10}


%\setlength{\bibsep}{0.3pt}
\setlength{\textfloatsep}{5pt}
\hypersetup{breaklinks=true,hypertexnames=false,colorlinks=true,citecolor = teal}
\captionsetup{font=normalsize}
\newcommand{\cmark}{\ding{51}}
\def\sym#1{\ifmmode^{#1}\else\(^{#1}\)\fi}
\renewcommand{\thetable}{\Roman{table}}
\geometry{verbose,tmargin=.9in,bmargin=1in,lmargin=.8in,rmargin=.8in,nomarginpar}
\makeatletter
\DeclareTextSymbolDefault{\textquotedbl}{T1}
\theoremstyle{plain}
\newtheorem{thm}{\protect\theoremname}
\theoremstyle{plain}
\newtheorem{prop}[thm]{\protect\propositionname}
\theoremstyle{definition}  % Add this line
\newtheorem{definition}[thm]{Definition}  % Add this line
\theoremstyle{remark}  % Add this line
\newtheorem{remark}[thm]{Remark}  % Add this line
\providecommand{\propositionname}{Proposition}
\providecommand{\theoremname}{Theorem}
\makeatother
\newtheorem{ass}[thm]{Assumption}
% \input{tcilatex}
\usepackage{tikz}
\usetikzlibrary{shapes.geometric, arrows, positioning}


\addbibresource{../references.bib}
\begin{document}

 
 This section summarizes Sharpe's (1990) original model. There is a continuum of firms (not necessarily risk-neutral) who all want to carry out a sequence of two investment projects. For each firm, the project at time $t=1,2$ transforms an investment of $I^t$ at the beginning of the period ($I^t$ is chosen by the firm) into a random return at the end of the period. This return depends on the firm's quality, $q$, and is given by
\[
\begin{cases}
X^t = g(I^t)I^t & \text{with probability } p_q, \\
0 & \text{with probability } 1-p_q,
\end{cases}
\]

where $g$ is strictly decreasing and concave. For each firm, returns for project 1 and 2 are stochastically independent, and the same is true across firms. There are two possible qualities of firms, $q=L,H$, with $p_L < p_H$. The proportion of high quality firms, $H$, is $\theta \in (0,1)$, and this is common knowledge. Firms do not know their own quality.
 
Because the key issue in Sharpe's (1990) paper is the problem of informational capture in relationship lending and because the variable investment case is a trivial extension, I assume from now on that project sizes $I^1$ and $I^2$ are fixed. Furthermore, to save on unnecessary indices, I consider the borrowing problem of one given firm, randomly drawn from the pool described above.

The firm has no own funds, but can borrow from competing banks. Banks are risk-neutral, compete \`{a} la Bertrand, and have unlimited access to funds at the net interest rate $\bar{r}$ per period. Like the firm, banks do not know the firm's quality at the beginning of period 1. However, if a bank finances the firm's first project, it perfectly observes the outcome of the project, which provides information about the firm's quality. Denote by
\[
y = 
\begin{cases} 
S & \text{if first period result is } X^1, \\
F & \text{if first period result is } 0
\end{cases}
\]
the firm's performance in the first project. It is assumed that the first project is financed by at most one bank, which becomes the ``inside bank.'' ``Outside banks,'' who have not provided first round finance, each get an identical, costless noisy signal of $y$, $\tilde{y}$, defined by
\[
\text{Pr}(\tilde{y}=\tilde{S} \mid S) = \text{Pr}(\tilde{y}=\tilde{F} \mid F) = \frac{1+\phi}{2},
\]
with $0 \le \phi < 1$. In the limiting case of $\phi=1$ inside and outside banks are both perfectly informed about the first-period outcome; this case is trivial. If $\phi=0$, the outside banks do not observe anything. Sharpe (1990) considers both the case of $\tilde{y}$ being observed by the inside bank (his Proposition 2) and of $\tilde{y}$ not being observed by the inside bank (Proposition 1). Because the case of unobserved $\tilde{y}$ may be less intuitive, I focus here on the latter case.

The key assumption concerning the strategic interaction among the players is the absence of binding long-term contracting possibilities. As forcefully argued by Sharpe (1990), this absence of long-term contracts is the interesting scenario to consider: without it the analysis would reduce to standard competitive pricing and miss the important point in bank relationships. The dynamic game played between the firm and the banks then has the following structure:

$t=1$: 1. Each bank $j$ announces a short-term lending rate $r_j^1$. \\
\indent 2. The firm chooses one bank, borrows and invests $I^1$. \\
\indent 3. The firm repays $(1+r^1)I^1$ iff $y=S$. Outside banks observe $\tilde{y}$. 
$t=2$: 4. Simultaneously, the inside bank offers a second-period interest rate $r_i^2 = r_i^2(\gamma)$ and each outside bank $h$ offers a second-period interest rate $r_h^2 = r_h^2(\tilde{\gamma})$.
\indent 5. The firm chooses an offer and invests $I^2$. If indifferent, the firm stays with the inside bank.
\indent 6. The firm repays $(1+r^2)I^2$ iff the second project has been successful.

The presentation of this game is slightly different from the one in Sharpe (1990), but both games are identical (with the restriction to one firm in my version). Apart from the absence of long-term contracting possibilities, two other assumptions of the model are

1. The firm consumes any profit after the first period.
2. Outstanding debt after a failure of the first project is forgiven.

The first assumption excludes the possibility of using retained earnings for investment and signaling purposes in the second period. The second eliminates all contractual links between the two periods, in particular, firms can switch freely from one bank to another despite their credit history. While these assumptions are somewhat extreme, they are useful simplifications to highlight the role of intertemporal informational constraints in bank competition.
Before analysing the model, it is useful to introduce, just as Sharpe (1990), some benchmark loan rates and notation. Let

\begin{align}
p &= \theta p_H + (1-\theta)p_L, \tag{1} \\
p(S) &= \frac{\text{Pr}(\gamma = S \ \& \ \text{success in } t=2)}{\text{Pr}(\gamma = S)} = \frac{\theta p_H^2 + (1-\theta)p_L^2}{p}, \tag{2} \\
p(F) &= \frac{\text{Pr}(\gamma = F \ \& \ \text{success in } t=2)}{\text{Pr}(\gamma = F)} = \frac{\theta (1-p_H)p_H + (1-\theta)(1-p_L)p_L}{1-p} \tag{3}
\end{align}


denote the success probabilities of the firm's second project, if there is, respectively, no information about first-period performance (Eq. (1)), if the first-period outcome has been observed to be good (Eq. (2)), and if the first-period outcome has been observed to be bad (Eq. (3)).
\indent Similarly, by Bayes' rule, the success probabilities conditional on the noisy observation $\tilde{\gamma}$ are given by
\[
p(\tilde{S}) = \text{Pr}(\tilde{\gamma}=\tilde{S} \ \& \ X^2 \mid \tilde{\gamma}=\tilde{S}) = \frac{(1-\Phi)p+2\Phi p(S)p}{(1-\Phi)+2\Phi p}
\]
and
\[
p(\tilde{F}) = \text{Pr}(\tilde{\gamma}=\tilde{F} \ \& \ X^2 \mid \tilde{\gamma}=\tilde{F}) = \frac{(1-\Phi)p+2\Phi p(F)(1-p)}{(1-\Phi)+2\Phi(1-p)}.
\]

Using these probabilities, one can define hypothetical zero-profit loan rates in each of these five situations:
\begin{align}
1+r_p &= \frac{1+\bar{r}}{p}, & 1+r_S &= \frac{1+\bar{r}}{p(S)}, & 1+r_F &= \frac{1+\bar{r}}{p(F)}, \tag{4} \\
1+r_{\tilde{S}} &= \frac{1+\bar{r}}{p(\tilde{S})}, & 1+r_{\tilde{F}} &= \frac{1+\bar{r}}{p(\tilde{F})}. \tag{5}
\end{align}
Clearly,
\begin{equation}
r_S < r_{\tilde{S}} < r_p < r_{\tilde{F}} < r_F. \tag{6}
\end{equation}

The final assumption is that $(1+r_F)I^2 \leqslant X^2$, i.e., that second-period lending is profitable even if the firm is known to have failed in the first period.

The natural solution concept for this game is Perfect Bayesian Nash equilibrium. The interesting part of the analysis of this game is the bidding competition between banks in the second period (Propositions 1 and 2 in Sharpe, 1990). The firm's response to competing bids is completely mechanic, and bank competition in stage 1 is standard bidding under symmetric information for the informational rent to be reaped in $t=2$.

The bidding game in the second period is a Bayesian game whose information structure (i.e., players' types and priors) has been determined in period 1. Denote pure strategies of outside banks by $r_h = r_h(\tilde{\gamma})$, let $r_o = r_o(\tilde{\gamma}) = \min_h r_h(\tilde{\gamma})$, and denote a pure strategy of the inside bank by $r_i = r_i(\gamma)$.

\textbf{Proposition 1.} \textit{The bidding game in stage 4 has no Bayesian Nash equilibrium in pure strategies.}

\textbf{Proof.} The proof is by contradiction.
\indent (1) Suppose that $r_o(\tilde{\gamma}) < r_{\tilde{\gamma}}$ for $\tilde{\gamma}=\tilde{S}$ or $\tilde{\gamma}=\tilde{F}$. Such an offer attracts at best (if $r_o(\tilde{\gamma}) < r_i(S)$) the $S$- and the $F$-type firm. In this case, $\tilde{\gamma}$ is an unbiased estimator of




\vspace{2cm}


the firm's $\gamma$ and the winning outside bank would make a strictly positive expected loss. Contradiction.
\indent (2) Suppose that $r_i(S) > \max(r_o(\tilde{S}), r_o(\tilde{F}))$. By (1) and Eq. (6), a deviation by the inside bank to $\max(r_o(\tilde{S}), r_o(\tilde{F}))$ would raise expected profits on the $S$-type strictly above zero. Contradiction.
\indent (3) Suppose that $r_o(\tilde{\gamma}) < r_i(S) < r_F$ for $\tilde{\gamma} = \tilde{S}$ or $\tilde{\gamma} = \tilde{F}$. By (2), $\min(r_o(\tilde{S}), r_o(\tilde{F})) < r_i(S) \leqslant \max(r_o(\tilde{S}), r_o(\tilde{F}))$. Then, by the optimality of $r_i(S)$, we must have $r_i(S) = \max(r_o(\tilde{S}), r_o(\tilde{F}))$. Because $r_i(F) \leqslant r_i(S)$ (which is smaller than $r_F$) is impossible, $\max(r_o(\tilde{S}), r_o(\tilde{F}))$ attracts exactly the $F$-type firm in equilibrium. Contradiction to $\max(r_o(\tilde{S}), r_o(\tilde{F})) < r_F$.
\indent (4) Suppose that $r_i(S) \leqslant r_o(\tilde{\gamma}) < r_F$ for $\tilde{\gamma} = \tilde{S}$ or $\tilde{\gamma} = \tilde{F}$. Then the outside banks' bid attracts at most the $F$-type firm as a customer. Clearly, $r_i(F) \geqslant r_F (> r_o(\tilde{\gamma}))$ (otherwise, the inside bank would make an expected loss on the $F$-firm). Hence, the outside offer attracts exactly the $F$-firm and makes a strictly positive expected loss because $r_o(\tilde{\gamma}) < r_F$.
\indent (5) Points (3) and (4) imply either directly that $r_i(S) \geqslant r_F$ or that $r_o(\tilde{\gamma}) \geqslant r_F$ for $\tilde{\gamma} = \tilde{S}$ and $\tilde{\gamma} = \tilde{F}$. If the latter is true, the optimality of $r_i(S)$ again implies $r_i(S) \geqslant r_F$. Clearly, also $r_i(F) \geqslant r_F$.
\indent If $\min(r_o(\tilde{S}), r_o(\tilde{F})) > r_i(S)$, then the inside bank would do better with a bid of $r_i(S) + \varepsilon$ for $\varepsilon$ sufficiently small, because this would allow it to realize higher profits per loan without loosing customers.
\indent If $\min(r_o(\tilde{S}), r_o(\tilde{F})) = r_i(S)$, then any of the winning outside banks would do better with a bid of $r_i(S) - \varepsilon$ for $\varepsilon$ sufficiently small, because this would allow it to attract the $S$-firm, on which it makes a strictly positive expected profit given its information.
\indent Suppose, therefore, finally that $\min(r_o(\tilde{S}), r_o(\tilde{F})) < r_i(S)$. By (2), $\max(r_o(\tilde{S}), r_o(\tilde{F})) \geqslant r_i(S)$. Because of competition from the inside bank, $r_i(F) \leqslant \max(r_o(\tilde{S}), r_o(\tilde{F}))$ or $\max(r_o(\tilde{S}), r_o(\tilde{F})) = r_F$. Hence, the winning outside banks for the signal $\tilde{\gamma}$ with $r_o(\tilde{\gamma}) = \max(r_o(\tilde{S}), r_o(\tilde{F}))$ make no profit on their offer. Because $r_{\tilde{F}} < r_F \leqslant r_i(S)$, they would be strictly better off undercutting $r_i(S)$ slightly, thus attracting both types of the firm. $\square$

Proposition 1 shows that Sharpe's (1990) Proposition 1, in which he proposes two pure-strategy equilibria as the solutions of the bidding game, is wrong. The problem with his proof is that he correctly rules out a number of pure-strategy combinations, but not all of them, and then concludes that what is left must be an equilibrium.
\indent The outcome described in Proposition 1 is a classical example of the winner's curse familiar from Bertrand competition and auction theory: if an outside bank wins the bidding contest, it must take into account that its success is due to its bid being attractive, but also to the fact that the inside bank did not want to bid more aggressively. Hence, the very fact of winning contains information that a rational player must take into account. Typically, in such situations pure-strategy equilibria do not exist.
\indent The proof of Proposition 1 can easily be adapted to the case of discrete action spaces (where interest rates must be expressed in terms of a smallest unit), as long as the interest rate grid is not too coarse. Furthermore, if the interest rate grid is coarse, it is clear that pure-strategy equilibria cannot exist because each deterministic interest rate choice allows to make too high a profit. The non-existence problem is, therefore, more fundamental than



the simple open-set problem which causes non-existence in Bertrand competition or first-price auctions under complete information and which disappears with discretization.
\indent A similar argument to the one given above shows that Sharpe's (1990) Proposition 2, which deals with the case of $\tilde{\gamma}$ being observable by the inside bank, is incorrect, as well. Intuitively, the case of observable $\tilde{\gamma}$ may be easier to understand than the one of unobservable $\tilde{\gamma}$, which is why the explicit proof given here considers the latter case. If $\tilde{\gamma}$ is unobservable to the inside bank, inside and outside banks all observe a signal unknown to the other. Hence, it may seem that all banks can condition deterministically on their signal and thereby still obtain sufficient randomness to rule out deviations, as, for example, in Gibbons and Katz (1991) in the context of labor markets. Proposition 1 shows that this intuition is not correct: the fact that one competitor's action cannot be predicted by the others is not enough, the competitors need to randomize actively. In the case of $\tilde{\gamma}$ being observable to the inside bank, however, the inside bank faces no noise, and randomization is even more plausible a priori.

\textbf{2. Equilibrium}

To simplify notation and some of the calculations, I consider here the case of extreme informational asymmetry $\Phi = 0$, in which the outside banks have no information. The analysis of the more general case introduced above is a relatively straightforward extension. Furthermore, I suppose that there is only one outside bank, which simplifies the analysis, but still conveys the full intuition.

\textbf{Proposition 2.} \textit{The Bayesian game between the inside and one outside bank in stage 4 of the dynamic bank competition game with $\Phi = 0$ has a unique mixed-strategy equilibrium. The inside bank's equilibrium strategy is to offer $r(F) = r_F$ with certainty and is an atomless distribution on $[r_p, r_F]$ for $\gamma = S$, with density}
\begin{equation}
h_i^S(r) = \frac{p(S)(1+r_p) - (1+\bar{r})}{(p(S)(1+r) - (1+\bar{r}))^2}. \tag{7}
\end{equation}
\textit{The outside bank's equilibrium strategy has a point mass of $1 - p(S)$ at $r = r_F$ and an atomless distribution on $[r_p, r_F)$ with density}
\begin{equation}
h_o(r) = p(S)h_i^S(r). \tag{8}
\end{equation}

The proof of the proposition is given in Appendix A. The strategy of the proof is to first characterise equilibrium strategies assuming that they exist, which yields a unique characterization, and then verify that the behaviour found constitutes indeed a Nash equilibrium. It is therefore not necessary to invoke the existence theorem of Dasgupta and Maskin (1986).
\indent Proposition 2 shows that firms switch banks in equilibrium. In fact, a bad firm switches whenever it receives an offer, which occurs with probability $p(S)$, but even good firms switch occasionally, namely with probability $\int_{r_p}^{r_F} (1 - H_i^S(r))h_o(r) \, dr$. Proposition 2 also shows that in equilibrium, full competition is only effective for bad firms. In particular, the



inside bank always offers the zero-profit interest rate $r_F$ to bad firms. Yet, with probability $p(S)$ bad firms are even getting excessively favorable terms on their loan, which happens when the outside bank underbids the inside bank. In this case, the outside bank makes a strict loss on the loan. Yet, the bank is neither acting irrationally nor recklessly: in order to put up a limited degree of competition for the good risks it must optimally take into account the occasional flop. On average, the outside bank puts up the maximum competition possible and makes zero expected profits.

\indent By the same token, since competition by the outside bank is limited, the inside bank makes positive expected profits on the good risks. Since the inside bank is indifferent between the interest rates in the interval $[r_p, r_F]$, these profits are proportional to $(r_p - r_S)$, which can be interpreted as a measure of adverse selection in the market. In fact, as seen following Eq. (A.10) in Appendix A, overall the inside bank's expected profits on good risks (expectation taken over the outside bank's randomization) are given by $p(S)(r_p - r_S)$. Hence, they are proportional to the success probability of the good firm. This is reasonable, although Proposition 2 also exhibits a countervailing effect: the higher $p(S)$, the tougher the competition by the outside bank. Moreover, the inside bank's pricing strategy, as given by Eq. (7), is quite intuitive: because $h_i^S$ is decreasing, most of the pricing occurs at moderate profit levels (slightly above $r_p$), with occasional attempts to really squeeze the firm (prices up to $r_F$).

\indent Concerning the robustness of the model, Proposition 2 can be easily adapted to cover the case $(1+r_F)I^2 > X^2$, in which second-period lending is not profitable if the firm is known to have failed in the first period. In this case, the inside bank does not continue financing a failed firm and randomizes atomlessly over the range $[r_p, X^2)$ with a point mass at $X^2$ for the successful firm. The outside bank does not bid at all with some probability $\mu$, and with probability $1 - \mu$ it bids and randomizes atomlessly over the whole range $[r_p, X^2]$.





\section{Appendix} \label{sec:appendix}

\paragraph{Step 1}
$\ell_i^\gamma \ge r_\gamma$ for $\gamma\in\{S,F\}$.

\emph{Proof.} Otherwise profits would be negative. 

\paragraph{Step 2}
$\ell_o \ge r_p$.

\emph{Proof.} we know $r_f> r_p$, using step 1 $\ell_i^f \geq r_f> r_p$ hence any offer $r<r_p$ attracts at best both groups and at worse only the failures. Given that the cost is at best the pooling cost, the outside bank will not make offers lower than the pooling cost. 

\paragraph{Step 3}
$\ell_i^S \ge r_p+ \lambda$.

\emph{Proof.} Follows from Step~2. Any offer, by the inside bank, lower than that could be raised slightly without decreasing the probability of winning. 

\paragraph{Step 4}
$\hat u_o \geq u_i^S $.

\emph{Proof.} Suppose $\hat u_o < u_i^S $, then the inside bank makes zero expected profits on all offers $r(s) \in (\hat u_0, u_i^s]$, however by step 3 the inside bank makes strictly positive profits on the S-firm. 


\paragraph{Step 5}
$H_i^S$ is continuous on $[\ell_i^S,u_i^S)$.

\emph{Proof.} 
Suppose that there is a $\hat r \in [\ell_i^S,u_i^S)$ at which $H_i^S$ is discontinuous,
i.e., with
\[
H_i^S(\hat r^-) < H_i^S(\hat r).
\]
Then, by Eq.~(A.6), $P_o(\hat r^-)>P_o(\hat r)$, because
$p(S)(1+r)-(1+\bar r)>0$ on $[\ell_i^S,u_i^S)$ by Step~3\footnote{Given that in step 3 we have that the expected profits of the incumbent when selling to S are strictly positive, the outside bank can switch some probability from $\hat r$ to $\hat r - \epsilon$ and increase their profits. }.

By the right-hand continuity of $H_i^\gamma$, $\gamma\in\{S,F\}$, there is an $\varepsilon>0$
such that $H_o(\hat r^-)=H_o(r)$ is constant on $[\hat r,\hat r+\varepsilon]$.
Therefore, $P_i^S$ is continuous at $\hat r$ and strictly increasing on
$[\hat r,\hat r+\varepsilon]$. Hence, $H_i^S$ can have no mass on
$[\hat r,\hat r+\varepsilon]$, which implies that
$H_i^S(\hat r^-)=H_i^S(\hat r)$. Contradiction.

Note that the proof of Step~5 does not apply to $H_i^F$, because we do not know whether
the inside bank makes strictly positive profits on the $F$-firm.


\medskip
\paragraph{Step 6} $u_i^S \ge \ell_i^F$.

\emph{Proof.} 
Suppose that $u_i^S < \ell_i^F$. This implies that the inside bank never makes an offer
$r\in(u_i^S,\ell_i^F)$.

\smallskip
\noindent(a) Suppose that $u_i^S < \hat u_o$. Then $\hat H_o$ can have no mass on
$[u_i^S,\ell_i^F)$, because for every offer $r\in[u_i^S,\ell_i^F)$ the offer
$\tfrac12(r+\ell_i^F)$ would be strictly better for the outside banks. Then the (positive)
mass of $\hat H_o$ on $[u_i^S,u_o]$ lies on $[\ell_i^F, \hat u_o]$. In particular, $\hat H_o$ is continuous
at $u_i^S$\footnote{The cdf will be flat on the $[u_i^S,\ell_i^F)$ interval. }.



Consider the following deviation from $H_i^S$: let $\delta>0$ and $\varepsilon>0$ be
given and small. Let $M_\varepsilon$ be the mass of $H_i^S$ on
$[u_i^S-\varepsilon,u_i^S]$. The deviation strategy is identical to $H_i^S$ on
$[\ell_i^S,u_i^S-\varepsilon)$, has zero mass on
$[u_i^S-\varepsilon,\ell_i^F-\delta)$ and point mass $M_\varepsilon$ on $\ell_i^F-\delta$.
The expected net gain (given $\gamma=S$) from this deviation is not smaller than
\begin{equation}
M_\varepsilon\Big[
(1-\hat H_o(u_i^S))(\ell_i^F-u_i^S-\delta)
-\big(\hat H_o(u_i^S)-\hat H_o(u_i^S-\varepsilon)\big)u_i^S
\Big].
\tag{A.7}
\end{equation}

The first of the two terms in Eq.~(A.7) (which corresponds to the total gain from the
deviation) is strictly positive for $\delta$ sufficiently small. The second term
(corresponding to the total loss from the deviation) tends to $0$ for $\varepsilon\to0$
by the continuity of $H_o$ at $u_i^S$. Hence, the deviation is strictly profitable for
$\delta$ and $\varepsilon$ small enough.

\medskip
\noindent(b) Suppose that $u_i^S=\hat u_o$. Consider the following deviation from $\hat H_o$:
let $\delta>0$ and $\varepsilon>0$ be given and small. Let $N_\varepsilon$ be the mass of
$\hat H_o$ on $[\hat u_o-\varepsilon,\hat u_o]$. Move all mass of $[\hat u_o-\varepsilon,\hat u_o)$ to
$\ell_i^F-\delta$. Then the expected net gain from this deviation is not smaller than
\[
N_\varepsilon\Big[
\underbrace{(1-p)[(\ell_i^F-\delta) - (u_o+ \lambda)]}_{Gain from \gamma = F}
- p\big(H_i^S(\hat u_o)-H_i^S( \hat u_o-\varepsilon)\big)
\big(p(S)(1+ \hat u_o)-(1+\bar r)\big)
\Big]
\]
where the second term now tends to $0$ for $\varepsilon\to0$ by Step~5.

\vspace{2cm}
 

the firm's $\gamma$ and the winning outside bank would make a strictly positive expected loss. Contradiction.
\indent (2) Suppose that $r_i(S) > \max(r_o(\tilde{S}), r_o(\tilde{F}))$. By (1) and Eq. (6), a deviation by the inside bank to $\max(r_o(\tilde{S}), r_o(\tilde{F}))$ would raise expected profits on the $S$-type strictly above zero. Contradiction.
\indent (3) Suppose that $r_o(\tilde{\gamma}) < r_i(S) < r_F$ for $\tilde{\gamma} = \tilde{S}$ or $\tilde{\gamma} = \tilde{F}$. By (2), $\min(r_o(\tilde{S}), r_o(\tilde{F})) < r_i(S) \leqslant \max(r_o(\tilde{S}), r_o(\tilde{F}))$. Then, by the optimality of $r_i(S)$, we must have $r_i(S) = \max(r_o(\tilde{S}), r_o(\tilde{F}))$. Because $r_i(F) \leqslant r_i(S)$ (which is smaller than $r_F$) is impossible, $\max(r_o(\tilde{S}), r_o(\tilde{F}))$ attracts exactly the $F$-type firm in equilibrium. Contradiction to $\max(r_o(\tilde{S}), r_o(\tilde{F})) < r_F$.
\indent (4) Suppose that $r_i(S) \leqslant r_o(\tilde{\gamma}) < r_F$ for $\tilde{\gamma} = \tilde{S}$ or $\tilde{\gamma} = \tilde{F}$. Then the outside banks' bid attracts at most the $F$-type firm as a customer. Clearly, $r_i(F) \geqslant r_F (> r_o(\tilde{\gamma}))$ (otherwise, the inside bank would make an expected loss on the $F$-firm). Hence, the outside offer attracts exactly the $F$-firm and makes a strictly positive expected loss because $r_o(\tilde{\gamma}) < r_F$.
\indent (5) Points (3) and (4) imply either directly that $r_i(S) \geqslant r_F$ or that $r_o(\tilde{\gamma}) \geqslant r_F$ for $\tilde{\gamma} = \tilde{S}$ and $\tilde{\gamma} = \tilde{F}$. If the latter is true, the optimality of $r_i(S)$ again implies $r_i(S) \geqslant r_F$. Clearly, also $r_i(F) \geqslant r_F$.
\indent If $\min(r_o(\tilde{S}), r_o(\tilde{F})) > r_i(S)$, then the inside bank would do better with a bid of $r_i(S) + \varepsilon$ for $\varepsilon$ sufficiently small, because this would allow it to realize higher profits per loan without loosing customers.
\indent If $\min(r_o(\tilde{S}), r_o(\tilde{F})) = r_i(S)$, then any of the winning outside banks would do better with a bid of $r_i(S) - \varepsilon$ for $\varepsilon$ sufficiently small, because this would allow it to attract the $S$-firm, on which it makes a strictly positive expected profit given its information.
\indent Suppose, therefore, finally that $\min(r_o(\tilde{S}), r_o(\tilde{F})) < r_i(S)$. By (2), $\max(r_o(\tilde{S}), r_o(\tilde{F})) \geqslant r_i(S)$. Because of competition from the inside bank, $r_i(F) \leqslant \max(r_o(\tilde{S}), r_o(\tilde{F}))$ or $\max(r_o(\tilde{S}), r_o(\tilde{F})) = r_F$. Hence, the winning outside banks for the signal $\tilde{\gamma}$ with $r_o(\tilde{\gamma}) = \max(r_o(\tilde{S}), r_o(\tilde{F}))$ make no profit on their offer. Because $r_{\tilde{F}} < r_F \leqslant r_i(S)$, they would be strictly better off undercutting $r_i(S)$ slightly, thus attracting both types of the firm. $\square$

Proposition 1 shows that Sharpe's (1990) Proposition 1, in which he proposes two pure-strategy equilibria as the solutions of the bidding game, is wrong. The problem with his proof is that he correctly rules out a number of pure-strategy combinations, but not all of them, and then concludes that what is left must be an equilibrium.
\indent The outcome described in Proposition 1 is a classical example of the winner's curse familiar from Bertrand competition and auction theory: if an outside bank wins the bidding contest, it must take into account that its success is due to its bid being attractive, but also to the fact that the inside bank did not want to bid more aggressively. Hence, the very fact of winning contains information that a rational player must take into account. Typically, in such situations pure-strategy equilibria do not exist.
\indent The proof of Proposition 1 can easily be adapted to the case of discrete action spaces (where interest rates must be expressed in terms of a smallest unit), as long as the interest rate grid is not too coarse. Furthermore, if the interest rate grid is coarse, it is clear that pure-strategy equilibria cannot exist because each deterministic interest rate choice allows to make too high a profit. The non-existence problem is, therefore, more fundamental than
 


\noindent\textbf{Step 7, original.} $u_i^F \le u_o$.

\emph{Proof.} 
Suppose that $u_i^F>u_o$.

Then the inside bank when setting $r \in (u_0, u_i^F]$ never sells, making zero profit. Thus by the indifference principle, the outside bank makes zero expected profits on the F-firms.
By Steps~4 and~6, $\ell_i^F\ge u_o$.
This and the zero expected profits imply $\ell_i^F=r_F$.

Consider the following deviation from $H_o$: let $\varepsilon>0$ be given and small.
Let $L_\varepsilon$ be the mass of $H_o$ on $[u_o-\varepsilon,u_o]$. Consider the case where the outisde bank concentrates all the remaining mass $L_\varepsilon$ on $\tfrac12(r_F+u_i^F)=:\alpha$. Given that $\alpha > r_f$ the firm makes a profit if chosen and since $\alpha < u_i^F$, there is a probability that the firm is chosen. Hence the firm makes strictly positive profits.

The expected net gain from this deviation is not smaller than
\[
L_\varepsilon\Big[
(1-p)\bigl(1-H_i^F(\alpha)\bigr)p(F)(\alpha-r_F)
- \underbrace{p\bigl(1-H_i^S(r_F-\varepsilon)\bigr)}_{\rightarrow 0 \text{ for } \varepsilon \to 0 \text{ by step 5}}
\big(p(S)(1+r_F)-(1+\bar r)\big)
\Big],
\]
 
Intuitively, since mass is taken away below $r_F$, the outside bank only gains if
$\gamma=F$.


\medskip 

 

\medskip
\noindent\textbf{Step 8. original} $u_i^F=r_F$.

\noindent\textbf{Proof.}
Clearly, $u_i^F\ge r_F$.\footnote{If $u_i^F < r_F$ the firm looses money on the F-firms, and if $u_i^F > r_F$ then } Suppose that $u_i^F>r_F$. Since the outside bank can obtain
strictly positive expected profits by choosing
$r=\tfrac12(r_F+u_i^F)$, it must make strictly positive profits also with $H_o$.
By Step~7, $u_o>u_i^F$ and by assumption $u_i^F > r_F$hence $u_o> r_F$ ; hence, also $H_i^F$ must make strictly positive expected profits\footnote{Since the insider knows the type can always bid in $u_0- \varepsilon$ and win in some cases a positive profit. }.
By Steps~4 and~7\footnote{By step 4 and 7 $u_0$ is higher than any bid made by the insider, hence always looses. Given that profits in $u_0$ are 0 then they are in the whole support. Note that this is because strategies have no atoms at the top because of undercutting. }, then,
\[
P_o(u_o)=0
\;\Rightarrow\;\footnotemark
H_o(u_o^-)=1 
\;\Rightarrow\;\footnotemark
P_i^S(u_o)=P_i^F(u_o)=0
\Rightarrow\;\footnotemark
H_i^S(u_o^-)=H_i^F(u_o^-)=1
\;\Rightarrow\;\footnotemark
P_o(u_o^-)=0,
\]
\footnotetext{Given that profits are strictly positive for $H_o$ and that at the top there are profits, there is no probability of playing $u_o$, by indifference among elements in the support.}
\footnotetext{Given that the outsider always plays lower than $u_o$, when the insider plays $u_o$ it never sells, hence profits are 0. }
\footnotetext{Given that with $u_o$ the insider makes zero profits and that he makes positive profits with other bids, then $u_o$ cannot be in the support of the insider. }
\footnotetext{The outside makes zero porfits because never sells at that price. }
which is a contradiction to the finding that $H_o$ makes strictly positive expected profits.
Step~8 implies that $r_i(F)=r_F$ with probability~1; in particular, the inside bank makes zero expected profits on the $F$-firm.

\medskip


\medskip
\noindent\textbf{Step 9.original} $u_o=u_i^S=r_F$.

\noindent\textbf{Proof.} Suppose that $u_o>r_F$. Then choosing $r_i(F)=\tfrac12(r_F+u_o)$ with probability~1 would yield the inside bank strictly positive expected profits on the $F$-firm\footnote{In step 8 was shown that it makes zero profit on the F-firm}.

The equality for $u_i^S$ follows from Steps~4 ($u_o \geq u_i^s$) and~6 ($u_i^s \geq \ell_i^F$), which imply that  $u_o \geq u_i^s \geq \ell_i^F$. We can use the part 1 of this same step to replace $u_o$ with $r_F$ to get $r_F \geq u_i^s \geq \ell_i^F$, moreover from step 8 we know that the F-strategy is only a point of mass: $ \ell_i^F = u_i^F = r_F$ hence $u_i^s = r_F$.


\medskip
\paragraph{Step 10.original}
The outside bank makes zero expected profits.

\noindent\textbf{Proof.}
By Steps~8 and~9\footnote{When playing $u_o=r_F$ the outside bank receives the F-firms and makes zero profits, hence it makes zero profits in the whole support.}, Eq.~(A.6) simplifies to
\begin{equation}
P_o(r)
=
p\bigl(1-H_i^S(r)\bigr)\bigl[p(S)(1+r)-(1+\bar r)\bigr]
+(1-p)\bigl[p(F)(1+r)-(1+\bar r)\bigr],
\tag{A.8}
\end{equation}
on $[\ell_o,u_o)$. By Step~5, $P_o$ is continuous on $[\ell_o,u_o)$, and by Step~9 we have
$P_o(u_o^-)=0$.




\medskip
\noindent\textbf{Step 11.original} $\ell_o=\ell_i^S=r_p$.

\noindent\textbf{Proof.}
It is impossible that $\ell_o>\ell_i^S$, because then the inside bank would make strictly higher profits if it placed the mass of $[\ell_i^S,\ell_o]$ on $\ell_o$. By a similar argument for the outside bank, $\ell_o<\ell_i^S$ is impossible. Finally, if $\ell_o>r_p$, the outside bank would make strictly positive expected profits, contradicting Step~10.




\paragraph{Step 12. original} $H_{o}$ is continuous on $[r_{p},r_{F})$.

\noindent\textbf{Proof.} Suppose that $H_{o}(\hat{r}^{-})<H_{o}(\hat{r})$ for some $\hat{r}\in[r_{p},r_{F})$. Then $P_{i}^{S}(\hat{r})>P_{i}^{S}(r)$ for $r\in(\hat{r},\hat{r}+\epsilon)$, $\epsilon>0$ sufficiently small, by the right-hand continuity of $H_{o}$. Hence, $H_{i}^{S}$ is constant on $(\hat{r},\hat{r}+\epsilon)$. Therefore, by Eq. (A.8), $P_{o}$ is strictly increasing on $(\hat{r}, \hat{r}+\epsilon)$. By the continuity of $P_{o}$ (which follows from Step 5 and Eq. (A.8)), $P_{o}(\hat{r})<P_{o}(\hat{r}+\epsilon)$. Hence, $H_{o}$ can have no mass on $\hat{r}$.

\paragraph{Step 13.original} $H_{i}^{S}$ and $H_{o}$ are strictly increasing on $[r_{p},r_{F}]$.

\noindent\textbf{Proof.} Suppose that $H_{i}^{S}$ is constant on some interval $[\alpha,b]\subset[r_{p},r_{F}].$ Let $[a,b]\supseteq[\alpha,b]$ be the maximal such interval. By Step 5 and the definition of $l_{i}^{S}$, $a>r_{p}$. Then $P_{o}$ is strictly increasing on $[a, b]$, hence, $H_{o}$ constant on $[a, b)$. By the continuity of $H_{o}$, $P_{i}^{S}$ is strictly increasing on $[a, b]$, a contradiction to the maximality of $[a, b]$.


The last step has completed the characterization of the mixed strategies, because it implies that $P_{i}^{S}$ and $P_{o}$ are constant on $[r_{p},r_{F}]$. By the continuity of $H_{o}$ and $H_{i}^{S}$ on $[r_{p},r_{F})$, we obtain therefore from Eqs. (A.5) and (A.8) for $r\in[r_{p},r_{F})$
\begin{align}
(1-H_{o}(r))[p(S)(1+r)-(1+\bar{r})] &= c, \tag{A.9}\\
p(1-H_{i}^{S}(r))[p(S)(1+r)-(1+\bar{r})] + (1-p)[p(F)(1+r)-(1+\bar{r})] &= 0. \tag{A.10}
\end{align}
The constant $c$ in Eq. (A.9) can be determined by substituting any value $r\in[r_{p},r_{F})$ into Eq. (A.9); for $r=r_{p}$ one obtains $c=p(S)(r_{p}-r_{S})$. Straightforward manipulations of Eqs. (A.9) and (A.10) then yield
\begin{align}
H_{i}^{S}(r) &= \frac{r-r_{p}}{p(S)(1+r)-(1+\bar{r})}, \tag{A.11}\\
H_{o}(r) &= p(S)\frac{r-r_{p}}{p(S)(1+r)-(1+\bar{r})} = p(S)H_{i}^{S}(r) \tag{A.12}
\end{align}
for $r\in[r_{p},r_{F})$. One easily checks that $H_{i}^{S}(r_{F}^{-})=1$, hence, $H_{i}^{S}$ is continuous on $[r_{p},r_{F}]$ and given by Eq. (A.11) on all of its domain. On the other hand, Eq. (A.12) shows that $H_{o}$ is discontinuous at $r=r_{F}$ with jump $1-p(S)$.

This identifies a unique mixed strategy profile. Because both players randomize over the whole of $[r_{p},r_{F}]$, there are no profitable deviations from these strategies for either player. Proposition 2 is therefore proved.


\end{document}